% pyformex manual --- olist
% $Id$
% (C) Benedict Verhegghe (benedict.verhegghe@ugent.be)
% DO NOT EDIT THIS FILE: it was automatically generated by 'gendoc.py'


\section{\module{olist} --- Some convenient shortcuts for common list operations.}
\label{sec:olist}

\declaremodule{""}{pyformex/olist}
\modulesynopsis{pyformex/olist}
\moduleauthor{'pyFormex project'}{'http://pyformex.org'}

While most of these functions look (and work) like set operations, their
result differs from using Python builtin Sets in that they preserve the
order of the items in the lists.


\subsection{Functions defined in module \module{olist}}

\begin{funcdesc}{union}{a,b}
Return a list with all items in a or in b, in the order of a,b.
\end{funcdesc}

\begin{funcdesc}{difference}{a,b}
Return a list with all items in a but not in b, in the order of a.
\end{funcdesc}

\begin{funcdesc}{symdifference}{a,b}
Return a list with all items in a or b but not in both.
\end{funcdesc}

\begin{funcdesc}{intersection}{a,b}
Return a list with all items in a and  in b, in the order of a.
\end{funcdesc}

\begin{funcdesc}{concatenate}{a}
Concatenate a list of lists
\end{funcdesc}

\begin{funcdesc}{flatten}{a,recurse=False}
Flatten a nested list.

    By default, lists are flattened one level deep.
    If recurse=True, flattening recurses through all sublists.
    
\end{funcdesc}

\begin{funcdesc}{select}{a,b}
Return a subset of items from a list.

    Returns a list with the items of a for which the index is in b.
    
\end{funcdesc}

%%% Local Variables: 
%%% mode: latex
%%% TeX-master: "pyformex"
%%% End:

