% pyformex manual --- section2d --- CREATED WITH py2ptex.py: DO NOT EDIT
% $Id$
% (C) B.Verhegghe

\section{\module{section2d} --- Some functions operating on 2D structures.}
\label{sec:section2d}

\declaremodule{""}{section2d}
\modulesynopsis{Some functions operating on 2D structures.}
\moduleauthor{'pyFormex project'}{'http://pyformex.berlios.de'}

This is a plugin for pyFormex.
(C) 2002 Benedict Verhegghe

See the Section2D example for an example of its use.



\subsection{planeSection class: A class describing a general 2D section.}

    The 2D section is the area inside a closed curve in the (x,y) plane.
    The curve is decribed by a finite number of points and by straight
    segments connecting them.
    


The planeSection class has this constructor: 

\begin{classdesc}{planeSection}{F}
Initialize a plane section.

        Initialization can be done either by a list of points or a set of line
        segments.

        1. By Points
        Each point is connected to the following one, and (unless they are
        very close) the last one back to the first. Traversing the resulting
        path should rotate positively around the z axis to yield a positive
        surface.

        2. By Segments
        It is the responsibilty of the user to ensure that the segments
        form a closed curve. If not, the calculated section data will be
        rather meaningless.
        

\end{classdesc}

planeSection objects have the following methods:

\begin{funcdesc}{sectionChar}{}

\end{funcdesc}


\subsection{Functions defined in the section2d module}



\begin{funcdesc}{sectionChar}{F}
Compute characteristics of plane sections.

    The plane sections are described by their circumference, consisting of a
    sequence of straight segments.
    The segment end point data are gathered in a plex-2 Formex.
    The segments should form a closed curve.
    The z-value of the coordinates does not have to be specified,
    and will be ignored if it is.
    The resulting path through the points should rotate positively around the
    z axis to yield a positive surface.

    The return value is a dict with the following characteristics:
    'L'   : circumference,
    'A'   : enclosed surface,
    'Sx'  : first area moment around global x-axis
    'Sy'  : first area moment around global y-axis
    'Ixx' : second area moment around global x-axis
    'Iyy' : second area moment around global y-axis
    'Ixy' : product moment of area around global x,y-axes
    

\end{funcdesc}


\begin{funcdesc}{extendedSectionChar}{S}
Computes extended section characteristics for the given section.

    S is a dict with section basic section characteristics as returned by
    sectionChar().
    This function computes and reutrns a dict with the following:
    'xG', 'yG' : coordinates of the center of gravity G of the plane section
    'IGxx', 'IGyy', 'IGxy' : second area moments and product around axes
       through G and  parallel with the global x,y-axes
    'alpha' : angle(in radians) between the glabla x,y axes and the principal
       axes (X,Y) of the section (X and Y always pass through G)
    'IXX','IYY': principal second area moments around X,Y respectively. (The
       second area product is always zero.)
    

\end{funcdesc}


\begin{funcdesc}{princTensor2D}{Ixx,Iyy,Ixy}
Compute the principal values and directions of a 2D tensor.

    Returns a tuple with three values:
    - alpha: angle (in radians) from x-axis to principal X-axis
    - IXX,IYY: principal values of the tensor
    

\end{funcdesc}


%%% Local Variables: 
%%% mode: latex
%%% TeX-master: "pyformex"
%%% End:

