% pyformex manual --- surface --- CREATED WITH py2ptex.py: DO NOT EDIT
% $Id$
% (C) B.Verhegghe

\section{\module{surface} --- Import/Export Formex structures to/from stl format.}
\label{sec:surface}

\declaremodule{""}{surface}
\modulesynopsis{Import/Export Formex structures to/from stl format.}
\moduleauthor{'pyFormex project'}{'http://pyformex.berlios.de'}

An stl is stored as a numerical array with shape [n,3,3].
This is compatible with the pyFormex data model.



\subsection{TriSurface class: A class for handling triangulated 3D surfaces.}




The TriSurface class has this constructor: 

\begin{classdesc}{TriSurface}{}
Create a new surface.

        The surface contains ntri triangles, each having 3 vertices with
        3 coordinates.
        The surface can be initialized from one of the following:
        - a (ntri,3,3) shaped array of floats ;
        - a 3-plex Formex with ntri elements ;
        - an (ncoords,3) float array of vertex coordinates and
          an (ntri,3) integer array of vertex numbers ;
        - an (ncoords,3) float array of vertex coordinates,
          an (nedges,2) integer array of vertex numbers,
          an (ntri,3) integer array of edges numbers.

        Internally, the surface is stored in a (coords,edges,faces) tuple.
        

\end{classdesc}

TriSurface objects have the following methods:

\begin{funcdesc}{getElems}{}
Get the elems data.
\end{funcdesc}

\begin{funcdesc}{setElems}{elems}
Change the elems data.
\end{funcdesc}

\begin{funcdesc}{refresh}{}
Make the internal information consistent and complete.

        This function should be called after one of the data fields
        have been changed.
        
\end{funcdesc}

\begin{funcdesc}{compress}{}
Remove all nodes which are not used.

        Normally, the surface definition can hold nodes that are not
        used in the edge/facet tables. They do however influence the
        bounding box of the surface.
        This method will remove all the unconnected nodes.
        
\end{funcdesc}

\begin{funcdesc}{append}{S}
Merge another surface with self.

        This just merges the data sets, and does not check
        whether the surfaces intersect or are connected!
        This is intended mostly for use inside higher level functions.
        
\end{funcdesc}

\begin{funcdesc}{ncoords}{}

\end{funcdesc}

\begin{funcdesc}{nedges}{}

\end{funcdesc}

\begin{funcdesc}{nfaces}{}

\end{funcdesc}

\begin{funcdesc}{nplex}{}

\end{funcdesc}

\begin{funcdesc}{ndim}{}

\end{funcdesc}

\begin{funcdesc}{vertices}{}

\end{funcdesc}

\begin{funcdesc}{shape}{}
Return the number of ;points, edges, faces of the TriSurface.
\end{funcdesc}

\begin{funcdesc}{copy}{}
Return a (deep) copy of the surface.

        If an index is given, only the specified faces are retained.
        
\end{funcdesc}

\begin{funcdesc}{select}{idx,compress=True}
Return a TriSurface which holds only elements with numbers in ids.

        idx can be a single element number or a list of numbers or
        any other index mechanism accepted by numpy's ndarray
        By default, the vertex list will be compressed to hold only those
        used in the selected elements.
        Setting compress==False will keep all original nodes in the surface.
        
\end{funcdesc}

\begin{funcdesc}{setProp}{p=None}
Create or delete the property array for the TriSurface.

        A property array is a rank-1 integer array with dimension equal
        to the number of elements in the TriSurface.
        You can specify a single value or a list/array of integer values.
        If the number of passed values is less than the number of elements,
        they wil be repeated. If you give more, they will be ignored.
        
        If a value None is given, the properties are removed from the TriSurface.
        
\end{funcdesc}

\begin{funcdesc}{prop}{}
Return the properties as a numpy array (ndarray)
\end{funcdesc}

\begin{funcdesc}{maxprop}{}
Return the highest property value used, or None
\end{funcdesc}

\begin{funcdesc}{propSet}{}
Return a list with unique property values.
\end{funcdesc}

\begin{funcdesc}{x}{}

\end{funcdesc}

\begin{funcdesc}{y}{}

\end{funcdesc}

\begin{funcdesc}{z}{}

\end{funcdesc}

\begin{funcdesc}{bbox}{}

\end{funcdesc}

\begin{funcdesc}{center}{}

\end{funcdesc}

\begin{funcdesc}{centroid}{}

\end{funcdesc}

\begin{funcdesc}{sizes}{}

\end{funcdesc}

\begin{funcdesc}{dsize}{}

\end{funcdesc}

\begin{funcdesc}{bsphere}{}

\end{funcdesc}

\begin{funcdesc}{centroids}{}
Return the centroids of all elements of the Formex.

        The centroid of an element is the point whose coordinates
        are the mean values of all points of the element.
        The return value is an (nfaces,3) shaped Coords array.
        
\end{funcdesc}

\begin{funcdesc}{distanceFromPlane}{None}

\end{funcdesc}

\begin{funcdesc}{distanceFromLine}{None}

\end{funcdesc}

\begin{funcdesc}{distanceFromPoint}{None}

\end{funcdesc}

\begin{funcdesc}{test}{nodes='all',dir=0,min=None,max=None}
Flag elements having nodal coordinates between min and max.

        This function is very convenient in clipping a TriSurface in a specified
        direction. It returns a 1D integer array flagging (with a value 1 or
        True) the elements having nodal coordinates in the required range.
        Use where(result) to get a list of element numbers passing the test.
        Or directly use clip() or cclip() to create the clipped TriSurface
        
        The test plane can be defined in two ways, depending on the value of dir.
        If dir == 0, 1 or 2, it specifies a global axis and min and max are
        the minimum and maximum values for the coordinates along that axis.
        Default is the 0 (or x) direction.

        Else, dir should be compaitble with a (3,) shaped array and specifies
        the direction of the normal on the planes. In this case, min and max
        are points and should also evaluate to (3,) shaped arrays.
        
        nodes specifies which nodes are taken into account in the comparisons.
        It should be one of the following:
        - a single (integer) point number (< the number of points in the Formex)
        - a list of point numbers
        - one of the special strings: 'all', 'any', 'none'
        The default ('all') will flag all the elements that have all their
        nodes between the planes x=min and x=max, i.e. the elements that
        fall completely between these planes. One of the two clipping planes
        may be left unspecified.
        
\end{funcdesc}

\begin{funcdesc}{clip}{t}
Return a TriSurface with all the elements where t>0.

        t should be a 1-D integer array with length equal to the number
        of elements of the TriSurface.
        The resulting TriSurface will contain all elements where t > 0.
        
\end{funcdesc}

\begin{funcdesc}{cclip}{t}
This is the complement of clip, returning a TriSurface where t<=0.
        
\end{funcdesc}

\begin{funcdesc}{pointNormals}{}
Compute the normal vectors in each point of a collection of triangles.
        
        The normal vector in a point is the average of the normal vectors of the neighbouring triangles.
        The normal vectors are normalized.
        
\end{funcdesc}

\begin{funcdesc}{offset}{distance=1.}
Offset a surface with a certain distance.
        
        All the nodes of the surface are translated over a specified distance along their normal vector.
        This creates a new congruent surface.
        
\end{funcdesc}

\begin{funcdesc}{toMesh}{}
Return a tuple of nodal coordinates and element connectivity.
\end{funcdesc}

\begin{funcdesc}{read}{clas,fn,ftype=None}
Read a surface from file.

        If no file type is specified, it is derived from the filename
        extension.
        Currently supported file types:
          - .stl (ASCII or BINARY)
          - .gts
          - .off
          - .neu (Gambit Neutral)
          - .smesh (Tetgen)
        
\classmethod
\end{funcdesc}

\begin{funcdesc}{write}{fname,ftype=None}
Write the surface to file.

        If no filetype is given, it is deduced from the filename extension.
        If the filename has no extension, the 'gts' file type is used.
        
\end{funcdesc}

\begin{funcdesc}{toFormex}{}
Convert the surface to a Formex.
\end{funcdesc}

\begin{funcdesc}{scale}{None}

\coordsmethod
\end{funcdesc}

\begin{funcdesc}{translate}{None}

\coordsmethod
\end{funcdesc}

\begin{funcdesc}{rotate}{None}

\coordsmethod
\end{funcdesc}

\begin{funcdesc}{shear}{None}

\coordsmethod
\end{funcdesc}

\begin{funcdesc}{reflect}{None}

\coordsmethod
\end{funcdesc}

\begin{funcdesc}{affine}{None}

\coordsmethod
\end{funcdesc}

\begin{funcdesc}{avgVertexNormals}{}
Compute the average normals at the vertices.
\end{funcdesc}

\begin{funcdesc}{areaNormals}{}
Compute the area and normal vectors of the surface triangles.

        The normal vectors are normalized.
        The area is always positive.

        The values are returned and saved in the object.
        
\end{funcdesc}

\begin{funcdesc}{facetArea}{}

\end{funcdesc}

\begin{funcdesc}{area}{}
Return the area of the surface
\end{funcdesc}

\begin{funcdesc}{volume}{}
Return the enclosed volume of the surface.

        This will only be correct if the surface is a closed manifold.
        
\end{funcdesc}

\begin{funcdesc}{edgeConnections}{}
Find the elems connected to edges.
\end{funcdesc}

\begin{funcdesc}{nodeConnections}{}
Find the elems connected to nodes.
\end{funcdesc}

\begin{funcdesc}{nEdgeConnected}{}
Find the number of elems connected to edges.
\end{funcdesc}

\begin{funcdesc}{nNodeConnected}{}
Find the number of elems connected to nodes.
\end{funcdesc}

\begin{funcdesc}{edgeAdjacency}{}
Find the elems adjacent to elems via an edge.
\end{funcdesc}

\begin{funcdesc}{nEdgeAdjacent}{}
Find the number of adjacent elems.
\end{funcdesc}

\begin{funcdesc}{nodeAdjacency}{}
Find the elems adjacent to elems via one or two nodes.
\end{funcdesc}

\begin{funcdesc}{nNodeAdjacent}{}
Find the number of adjacent elems.
\end{funcdesc}

\begin{funcdesc}{surfaceType}{}

\end{funcdesc}

\begin{funcdesc}{borderEdges}{}
Detect the border elements of TriSurface.

        The border elements are the edges having less than 2 connected elements.
        Returns True where edge is on the border.
        
\end{funcdesc}

\begin{funcdesc}{borderEdgeNrs}{}
Returns the numbers of the border edges.
\end{funcdesc}

\begin{funcdesc}{borderNodeNrs}{}
Detect the border nodes of TriSurface.

        The border nodes are the vertices belonging to the border edges.
        Returns a list of vertex numbers.
        
\end{funcdesc}

\begin{funcdesc}{isManifold}{}

\end{funcdesc}

\begin{funcdesc}{isClosedManifold}{}

\end{funcdesc}

\begin{funcdesc}{checkBorder}{}
Return the border of TriSurface as a set of segments.
\end{funcdesc}

\begin{funcdesc}{fillBorder}{method=0}
If the surface has a single closed border, fill it.

        Filling the border is done by adding a single point inside
        the border and connectin it with all border segments.
        This works well if the border is smooth and nearly planar.
        
\end{funcdesc}

\begin{funcdesc}{border}{}
Return the border of TriSurface as a Plex-2 Formex.
\end{funcdesc}

\begin{funcdesc}{edgeCosAngles}{}
Return the cos of the angles over all edges.
        
        The surface should be a manifold (max. 2 elements per edge).
        Edges with only one element get angles = 1.0.
        
\end{funcdesc}

\begin{funcdesc}{edgeAngles}{}
Return the angles over all edges (in degrees).
\end{funcdesc}

\begin{funcdesc}{data}{}
Compute data for all edges and faces.
\end{funcdesc}

\begin{funcdesc}{aspectRatio}{}

\end{funcdesc}

\begin{funcdesc}{smallestAltitude}{}

\end{funcdesc}

\begin{funcdesc}{longestEdge}{}

\end{funcdesc}

\begin{funcdesc}{shortestEdge}{}

\end{funcdesc}

\begin{funcdesc}{stats}{}
Return a text with full statistics.
\end{funcdesc}

\begin{funcdesc}{edgeFront}{startat=0,okedges=None,front_increment=1}
Generator function returning the frontal elements.

        startat is an element number or list of numbers of the starting front.
        On first call, this function returns the starting front.
        Each next() call returns the next front.
        front_increment determines haw the property increases at each
        frontal step. There is an extra increment +1 at each start of
        a new part. Thus, the start of a new part can always be detected
        by a front not having the property of the previous plus front_increment.
        
\end{funcdesc}

\begin{funcdesc}{nodeFront}{startat=0,front_increment=1}
Generator function returning the frontal elements.

        startat is an element number or list of numbers of the starting front.
        On first call, this function returns the starting front.
        Each next() call returns the next front.
        
\end{funcdesc}

\begin{funcdesc}{walkEdgeFront}{startat=0,nsteps=-1,okedges=None,front_increment=1}

\end{funcdesc}

\begin{funcdesc}{walkNodeFront}{startat=0,nsteps=-1,front_increment=1}

\end{funcdesc}

\begin{funcdesc}{growSelection}{sel,mode='node',nsteps=1}
Grows a selection of a surface.

        p is a single element number or a list of numbers.
        The return value is a list of element numbers obtained by
        growing the front nsteps times.
        The mode argument specifies how a single frontal step is done:
        'node' : include all elements that have a node in common,
        'edge' : include all elements that have an edge in common.
        
\end{funcdesc}

\begin{funcdesc}{partitionByEdgeFront}{okedges,firstprop=0,startat=0}
Detects different parts of the surface using a frontal method.

        okedges flags the edges where the two adjacent triangles are to be
        in the same part of the surface.
        startat is a list of elements that are in the first part. 
        The partitioning is returned as a property type array having a value
        corresponding to the part number. The lowest property number will be
        firstprop
        
\end{funcdesc}

\begin{funcdesc}{partitionByNodeFront}{firstprop=0,startat=0}
Detects different parts of the surface using a frontal method.

        okedges flags the edges where the two adjacent triangles are to be
        in the same part of the surface.
        startat is a list of elements that are in the first part. 
        The partitioning is returned as a property type array having a value
        corresponding to the part number. The lowest property number will be
        firstprop
        
\end{funcdesc}

\begin{funcdesc}{partitionByConnection}{}

\end{funcdesc}

\begin{funcdesc}{partitionByAngle}{angle=180.,firstprop=0,startat=0}

\end{funcdesc}

\begin{funcdesc}{cutWithPlane}{None}
Cut a surface with a plane.
\end{funcdesc}

\begin{funcdesc}{connectedElements}{target,elemlist=None}
Return the elements from list connected with target
\end{funcdesc}

\begin{funcdesc}{smoothLowPass}{n_iterations=2,lambda_value=0.5}
Smooth the surface using a low-pass filter.
\end{funcdesc}

\begin{funcdesc}{smoothLaplaceHC}{n_iterations=2,lambda_value=0.5,alpha=0.,beta=0.2}
Smooth the surface using a Laplace filter and HC algorithm.
\end{funcdesc}

\begin{funcdesc}{check}{verbose=False}
Check the surface using gtscheck.
\end{funcdesc}

\begin{funcdesc}{split}{base,verbose=False}
Check the surface using gtscheck.
\end{funcdesc}

\begin{funcdesc}{coarsen}{min_edges=None,max_cost=None,mid_vertex=False,length_cost=False,max_fold=1.0,volume_weight=0.5,boundary_weight=0.5,shape_weight=0.0,progressive=False,log=False,verbose=False}
Coarsen the surface using gtscoarsen.
\end{funcdesc}

\begin{funcdesc}{refine}{max_edges=None,min_cost=None,log=False,verbose=False}
Refine the surface using gtsrefine.
\end{funcdesc}

\begin{funcdesc}{smooth}{lambda_value=0.5,n_iterations=2,fold_smoothing=None,verbose=False}
Smooth the surface using gtssmooth.
\end{funcdesc}

\begin{funcdesc}{boolean}{surf,op,inter=False,check=False,verbose=False}
Perform a boolean operation with surface surf.

        
\end{funcdesc}


\subsection{Functions defined in the surface module}



\begin{funcdesc}{areaNormals}{x}
Compute the area and normal vectors of a collection of triangles.

    x is an (ntri,3,3) array of coordinates.

    Returns a tuple of areas,normals.
    The normal vectors are normalized.
    The area is always positive.
    

\end{funcdesc}


\begin{funcdesc}{stlConvert}{stlname,outname=None,options='-d'}
Transform an .stl file to .off or .gts format.

    If outname is given, it is either '.off' or '.gts' or a filename ending
    on one of these extensions. If it is only an extension, the stlname will
    be used with extension changed.

    If the outname file exists and its mtime is more recent than the stlname,
    the outname file is considered uptodate and the conversion programwill
    not be run.
    
    The conversion program will be choosen depending on the extension.
    This uses the external commands 'admesh' or 'stl2gts'.

    The return value is a tuple of the output file name, the conversion
    program exit code (0 if succesful) and the stdout of the conversion
    program (or a 'file is already uptodate' message).
    

\end{funcdesc}


\begin{funcdesc}{read_gts}{fn}
Read a GTS surface mesh.

    Return a coords,edges,faces tuple.
    

\end{funcdesc}


\begin{funcdesc}{read_off}{fn}
Read an OFF surface mesh.

    The mesh should consist of only triangles!
    Returns a nodes,elems tuple.
    

\end{funcdesc}


\begin{funcdesc}{read_stl}{fn,intermediate=None}
Read a surface from .stl file.

    This is done by first coverting the .stl to .gts or .off format.
    The name of the intermediate file may be specified. If not, it will be
    generated by changing the extension of fn to '.gts' or '.off' depending
    on the setting of the 'surface/stlread' config setting.
    
    Return a coords,edges,faces or a coords,elems tuple, depending on the
    intermediate format.
    

\end{funcdesc}


\begin{funcdesc}{read_gambit_neutral}{fn}
Read a triangular surface mesh in Gambit neutral format.

    The .neu file nodes are numbered from 1!
    Returns a nodes,elems tuple.
    

\end{funcdesc}


\begin{funcdesc}{write_gts}{fn,nodes,edges,faces}


\end{funcdesc}


\begin{funcdesc}{write_stla}{f,x}
Export an x[n,3,3] float array as an ascii .stl file.

\end{funcdesc}


\begin{funcdesc}{write_stlb}{f,x}
Export an x[n,3,3] float array as an binary .stl file.

\end{funcdesc}


\begin{funcdesc}{write_gambit_neutral}{fn,nodes,elems}


\end{funcdesc}


\begin{funcdesc}{write_off}{fn,nodes,elems}


\end{funcdesc}


\begin{funcdesc}{write_smesh}{fn,nodes,elems}


\end{funcdesc}


\begin{funcdesc}{surface_volume}{x,pt=None}
Return the volume inside a 3-plex Formex.

    For each element of Formex, return the volume of the tetrahedron
    formed by the point pt (default the center of x) and the 3 points
    of the element.
    

\end{funcdesc}


\begin{funcdesc}{surfaceInsideLoop}{coords,elems}
Create a surface inside a closed curve defined by coords and elems.

    coords is a set of coordinates.
    elems is an (nsegments,2) shaped connectivity array defining a set of line
    segments forming a closed loop.

    The return value is coords,elems tuple where
    coords has one more point: the center of th original coords
    elems is (nsegment,3) and defines triangles describing a surface inside
    the original curve.
    

\end{funcdesc}


\begin{funcdesc}{fillHole}{coords,elems}
Fill a hole surrounded by the border defined by coords and elems.
    
    Coords is a (npoints,3) shaped array of floats.
    Elems is a (nelems,2) shaped array of integers representing the border
    element numbers and must be ordered.
    

\end{funcdesc}


\begin{funcdesc}{create_border_triangle}{coords,elems}
Create a triangle within a border.
    
    The triangle is created from the two border elements with
    the sharpest angle.
    Coords is a (npoints,3) shaped array of floats.
    Elems is a (nelems,2) shaped array of integers representing
    the border element numbers and must be ordered.
    A list of two objects is returned: the new border elements and the triangle.
    

\end{funcdesc}


\begin{funcdesc}{read_error}{cnt,line}
Raise an error on reading the stl file.

\end{funcdesc}


\begin{funcdesc}{degenerate}{area,norm}
Return a list of the degenerate faces according to area and normals.

    A face is degenerate if its surface is less or equal to zero or the
    normal has a nan.
    

\end{funcdesc}


\begin{funcdesc}{read_stla}{fn,dtype=Float,large=False,guess=True}
Read an ascii .stl file into an [n,3,3] float array.

    If the .stl is large, read_ascii_large() is recommended, as it is
    a lot faster.
    

\end{funcdesc}


\begin{funcdesc}{read_ascii_large}{fn,dtype=Float}
Read an ascii .stl file into an [n,3,3] float array.

    This is an alternative for read_ascii, which is a lot faster on large
    STL models.
    It requires the 'awk' command though, so is probably only useful on
    Linux/UNIX. It works by first transforming  the input file to a
    .nodes file and then reading it through numpy's fromfile() function.
    

\end{funcdesc}


\begin{funcdesc}{off_to_tet}{fn}
Transform an .off model to tetgen (.node/.smesh) format.

\end{funcdesc}


\begin{funcdesc}{find_row}{mat,row,nmatch=None}
Find all rows in matrix matching given row.

\end{funcdesc}


\begin{funcdesc}{find_nodes}{nodes,coords}
Find nodes with given coordinates in a node set.

    nodes is a (nnodes,3) float array of coordinates.
    coords is a (npts,3) float array of coordinates.

    Returns a (n,) integer array with ALL the node numbers matching EXACTLY
    ALL the coordinates of ANY of the given points.
    

\end{funcdesc}


\begin{funcdesc}{find_first_nodes}{nodes,coords}
Find nodes with given coordinates in a node set.

    nodes is a (nnodes,3) float array of coordinates.
    coords is a (npts,3) float array of coordinates.

    Returns a (n,) integer array with THE FIRST node number matching EXACTLY
    ALL the coordinates of EACH of the given points.
    

\end{funcdesc}


\begin{funcdesc}{find_triangles}{elems,triangles}
Find triangles with given node numbers in a surface mesh.

    elems is a (nelems,3) integer array of triangles.
    triangles is a (ntri,3) integer array of triangles to find.
    
    Returns a (ntri,) integer array with the triangles numbers.
    

\end{funcdesc}


\begin{funcdesc}{remove_triangles}{elems,remove}
Remove triangles from a surface mesh.

    elems is a (nelems,3) integer array of triangles.
    remove is a (nremove,3) integer array of triangles to remove.
    
    Returns a (nelems-nremove,3) integer array with the triangles of
    nelems where the triangles of remove have been removed.
    

\end{funcdesc}


\begin{funcdesc}{Rectangle}{nx,ny}
Create a plane rectangular surface consisting of a nx,ny grid.

\end{funcdesc}


\begin{funcdesc}{Cube}{}
Create a surface in the form of a cube

\end{funcdesc}


\begin{funcdesc}{Sphere}{level=4,verbose=False,filename=None}
Create a spherical surface by caling the gtssphere command.

    If a filename is given, it is stored under that name, else a temporary
    file is created.
    Beware: this may take a lot of time if level is 8 or higher.
    

\end{funcdesc}


%%% Local Variables: 
%%% mode: latex
%%% TeX-master: "pyformex"
%%% End:

