% pyformex manual --- project --- CREATED WITH py2ptex.py: DO NOT EDIT
% $Id$
% (C) B.Verhegghe

\section{\module{project} --- project.py}
\label{sec:project}

\declaremodule{""}{project}
\modulesynopsis{project.py}
\moduleauthor{'pyFormex project'}{'http://pyformex.berlios.de'}

Functions for managing a project in pyFormex.



\subsection{Project class: A project is a persistent storage of a Python dictionary.}




The Project class has this constructor: 

\begin{classdesc}{Project}{filename,create=False,signature=_signature_,compression=0,binary=False,legacy=True}
Create a new project with the given filename.

        If the filename exists and create is False, the file is opened and
        the contents is read into the project dictionary.
        If not, a new empty file and project are created.

        If legacy = True, the Project is allowed to read unsigned file formats.
        Writing is always done with signature though.
        

\end{classdesc}

Project objects have the following methods:

\begin{funcdesc}{header_data}{}
Construct the data to be saved in the header.
\end{funcdesc}

\begin{funcdesc}{set_data_from_header}{data}
Set the project data from the header.
\end{funcdesc}

\begin{funcdesc}{save}{}
Save the project to file.
\end{funcdesc}

\begin{funcdesc}{load}{}
Load a project from file.
        
        The loaded definitions will update the current project.
        
\end{funcdesc}

\begin{funcdesc}{delete}{}
Unrecoverably delete the project file.
\end{funcdesc}


%%% Local Variables: 
%%% mode: latex
%%% TeX-master: "pyformex"
%%% End:

