% pyformex manual --- colors --- CREATED WITH py2ptex.py: DO NOT EDIT
% $Id$
% (C) B.Verhegghe

\section{\module{colors} --- Definition of some RGB colors and color conversion functions}
\label{sec:colors}

\declaremodule{""}{colors}
\modulesynopsis{Definition of some RGB colors and color conversion functions}
\moduleauthor{'pyFormex project'}{'http://pyformex.berlios.de'}




\begin{funcdesc}{GLColor}{color}
Convert a color to an OpenGL RGB color.

    The output is a tuple of three RGB float values ranging from 0.0 to 1.0.
    The input can be any of the following:
    - a QColor
    - a string specifying the Xwindow name of the color
    - a hex string '\#RGB' with 1 to 4 hexadecimal digits per color 
    - a tuple or list of 3 integer values in the range 0..255
    - a tuple or list of 3 float values in the range 0.0..1.0
    Any other input may give unpredictable results.
    

\end{funcdesc}


\begin{funcdesc}{colorName}{color}
Return a string designation for the color.

    color can be anything that is accepted by GLColor.
    In most cases
    If color can not be converted, None is returned.
    

\end{funcdesc}


\begin{funcdesc}{createColorDict}{}


\end{funcdesc}


\begin{funcdesc}{closestColorName}{color}
Return the closest color name.

\end{funcdesc}


\begin{funcdesc}{RGBA}{rgb,alpha=1.0}
Adds an alpha channel to an RGB color

\end{funcdesc}


\begin{funcdesc}{GREY}{val,alpha=1.0}
Returns a grey OpenGL color of given intensity (0..1)

\end{funcdesc}


\begin{funcdesc}{grey}{i}


\end{funcdesc}


%%% Local Variables: 
%%% mode: latex
%%% TeX-master: "pyformex"
%%% End:

