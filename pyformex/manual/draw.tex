% pyformex manual --- draw
% $Id$
% (C) B.Verhegghe

\section{\module{draw} --- Create graphical representations}
\label{sec:drawing}

\emph{Warning: the \texttt{draw} module is still subject to regular changes. Therefore, the information given below may not be fully accurate.
}

\declaremodule{""}{draw}
\modulesynopsis{Create graphical representations of 3D geometries.}
\moduleauthor{'pyFormex project'}{'http://pyformex.berlios.de'}

The following functions provide access to the OpenGL 3D drawing capabilities of \pyformex.


\begin{funcdesc}{draw}{F, view=None,bbox='auto',
         color='prop',colormap=None,alpha=0.5,
         mode=None,linewidth=None,shrink=None,marksize=None,
         wait=True,clear=None,allviews=False)}

Draw object(s) with specified settings and direct camera to it.

The first argument is an object to be drawn. All other arguments are
settings that influence how  the object is being drawn.

F is either a Formex or a TriSurface object, or a name of such object
(global or exported), or a list thereof.
If F is a list, the draw() function is called repeatedly with each of
ithe items of the list as first argument and with the remaining arguments
unchanged.

view is either the name of a defined view or 'last' or None.
Predefined views are 'front','back','top','bottom','left','right','iso'.
With view=None the camera settings remain unchanged (but might be changed
interactively through the user interface). This may make the drawn object
out of view!
With view='last', the camera angles will be set to the same camera angles
as in the last draw operation, undoing any interactive changes.
The initial default view is 'front' (looking in the -z direction).

bbox specifies the 3D volume at which the camera will be aimed (using the
angles set by view). The camera position wil be set so that the volume
comes in view using the current lens (default 45 degrees).
bbox is a list of two points or compatible (array with shape (2,3)).
Setting the bbox to a volume not enclosing the object may make the object
invisible on the canvas.
The default bbox='auto' will use the bounding box of the objects getting
drawn (object.bbox()), thus ensuring that the camera will focus on the
shown object.
With bbox=None, the camera's target volume remains unchanged.

color,colormap,linewidth,alpha,marksize are passed to the
creation of the 3D actor.

shrink is a floating point shrink factor that will be applied to object
before drawing it.

If the Formex has properties and a color list is specified, then the
the properties will be used as an index in the color list and each member
will be drawn with the resulting color.
If color is one color value, the whole Formex will be drawn with
that color.
Finally, if color=None is specified, the whole Formex is drawn in black.

Each draw action activates a locking mechanism for the next draw action,
which will only be allowed after drawdelay seconds have elapsed. This
makes it easier to see subsequent images and is far more elegant that an
explicit sleep() operation, because all script processing will continue
up to the next drawing instruction.
The value of drawdelay is set in the config, or 2 seconds by default.
The user can disable the wait cycle for the next draw operation by
specifying wait=False. Setting drawdelay=0 will disable the waiting
mechanism for all subsequent draw statements (until set >0 again).

\end{funcdesc}

\begin{funcdesc}{drawwait}{}
Wait for the drawing lock to be released.

While we are waiting, events are processed.

\end{funcdesc}

\begin{funcdesc}{drawlock}{}
Lock the drawing function for the next drawdelay seconds.
\end{funcdesc}

\begin{funcdesc}{drawblock}{}
Lock the drawing function indefinitely.
\end{funcdesc}

\begin{funcdesc}{drawrelease}{}
Release the drawing function.

If a timer is running, cancel it.

\end{funcdesc}

\begin{funcdesc}{reset}{}
\end{funcdesc}

\begin{funcdesc}{resetAll}{}
\end{funcdesc}

\begin{funcdesc}{setDrawOptions}{d}
\end{funcdesc}

\begin{funcdesc}{showDrawOptions}{}
\end{funcdesc}

\begin{funcdesc}{shrink}{v}
\end{funcdesc}

\begin{funcdesc}{setView}{name,angles=None}
Set the default view for future drawing operations.

If no angles are specified, the name should be an existing view, or
the predefined value '__last__'.
If angles are specified, this is equivalent to createView(name,angles)
followed by setView(name).

\end{funcdesc}

\begin{funcdesc}{_shrink}{F,factor}
Return a shrinked object.

A shrinked object is one where each element is shrinked with a factor
around its own center.

\end{funcdesc}

\begin{funcdesc}{drawPlane}{P,N,size}
\end{funcdesc}

\begin{funcdesc}{drawNumbers}{F,color=colors.black,trl=None,offset=0}
Draw numbers on all elements of F.

Normally, the numbers are drawn at the centroids of the elements.
A translation may be given to put the numbers out of the centroids,
e.g. to put them in front of the objects to make them visible,
or to allow to view a mark at the centroids.

\end{funcdesc}

\begin{funcdesc}{drawVertexNumbers}{F,color=colors.black,trl=None}
Draw (local) numbers on all vertices of F.

Normally, the numbers are drawn at the location of the vertices.
A translation may be given to put the numbers out of the location,
e.g. to put them in front of the objects to make them visible,
or to allow to view a mark at the vertices.

\end{funcdesc}

\begin{funcdesc}{drawText3D}{P,text,color=colors.black,font=None}
Draw a text at a 3D point.
\end{funcdesc}

\begin{funcdesc}{drawViewportAxes3D}{pos,color=None}
Draw two viewport axes at a 3D position.
\end{funcdesc}

\begin{funcdesc}{drawBbox}{A}
Draw the bbox of the actor A.
\end{funcdesc}

\begin{funcdesc}{drawActor}{A}
Draw an actor and update the screen.
\end{funcdesc}

\begin{funcdesc}{undraw}{itemlist}
Remove an item or a number of items from the canvas.

Use the return value from one of the draw... functions to remove
the item that was drawn from the canvas.
A single item or a list of items may be specified.

\end{funcdesc}

\begin{funcdesc}{focus}{object}
Move the camera thus that object comes fully into view.

object can be anything having a bbox() method.

The camera is moved with fixed axis directions to a place
where the whole object can be viewed using a 45. degrees lens opening.
This technique may change in future!

\end{funcdesc}

\begin{funcdesc}{view}{v,wait=False}
Show a named view, either a builtin or a user defined.
\end{funcdesc}

\begin{funcdesc}{setTriade}{on=None}
Toggle the display of the global axes on or off.

If on is True, the axes triade is displayed, if False it is
removed. The default (None) toggles between on and off.

\end{funcdesc}

\begin{funcdesc}{drawtext}{text,x,y,font='9x15',adjust='left',color=None}
Show a text at position x,y using font.
\end{funcdesc}

\begin{funcdesc}{annotate}{annot}
Draw an annotation.
\end{funcdesc}

\begin{funcdesc}{unannotate}{annot}
\end{funcdesc}

\begin{funcdesc}{decorate}{decor}
Draw a decoration.
\end{funcdesc}

\begin{funcdesc}{undecorate}{decor}
\end{funcdesc}

\begin{funcdesc}{frontView}{}
\end{funcdesc}

\begin{funcdesc}{backView}{}
\end{funcdesc}

\begin{funcdesc}{leftView}{}
\end{funcdesc}

\begin{funcdesc}{rightView}{}
\end{funcdesc}

\begin{funcdesc}{topView}{}
\end{funcdesc}

\begin{funcdesc}{bottomView}{}
\end{funcdesc}

\begin{funcdesc}{isoView}{}
\end{funcdesc}

\begin{funcdesc}{createView}{name,angles}
Create a new named view (or redefine an old).

The angles are (longitude, latitude, twist).
If the view name is new, and there is a views toolbar,
a view button will be added to it.

\end{funcdesc}

\begin{funcdesc}{zoomBbox}{bb}
Zoom thus that the specified bbox becomes visible.
\end{funcdesc}

\begin{funcdesc}{zoomAll}{}
Zoom thus that all actors become visible.
\end{funcdesc}

\begin{funcdesc}{zoom}{f}
\end{funcdesc}

\begin{funcdesc}{bgcolor}{color}
Change the background color (and redraw).
\end{funcdesc}

\begin{funcdesc}{fgcolor}{color}
Set the default foreground color.
\end{funcdesc}

\begin{funcdesc}{renderMode}{mode}
\end{funcdesc}

\begin{funcdesc}{wireframe}{}
\end{funcdesc}

\begin{funcdesc}{flat}{}
\end{funcdesc}

\begin{funcdesc}{smooth}{}
\end{funcdesc}

\begin{funcdesc}{smoothwire}{}
\end{funcdesc}

\begin{funcdesc}{flatwire}{}
\end{funcdesc}

\begin{funcdesc}{opacity}{alpha}
Set the viewports transparency.
\end{funcdesc}

\begin{funcdesc}{lights}{onoff}
Set the lights on or off
\end{funcdesc}

\begin{funcdesc}{set_light_value}{typ,val}
Set the value of one of the lighting parameters for the currrent view

typ is one of 'ambient','specular','emission','shininess'
val is a value between 0.0 and 1.0

\end{funcdesc}

\begin{funcdesc}{set_ambient}{i}
\end{funcdesc}

\begin{funcdesc}{set_specular}{i}
\end{funcdesc}

\begin{funcdesc}{set_emission}{i}
\end{funcdesc}

\begin{funcdesc}{set_shininess}{i}
\end{funcdesc}

\begin{funcdesc}{linewidth}{wid}
Set the linewidth to be used in line drawings.
\end{funcdesc}

\begin{funcdesc}{clear_canvas}{}
Clear the canvas.

This is a low level function not intended for the user.

\end{funcdesc}

\begin{funcdesc}{clear}{}
Clear the canvas
\end{funcdesc}

\begin{funcdesc}{redraw}{}
\end{funcdesc}

\begin{funcdesc}{pause}{msg="Use the Step/Continue buttons to proceed"}
\end{funcdesc}

\begin{funcdesc}{step}{}
Perform one step of a script.

A step is a set of instructions until the next draw operation.
If a script is running, this just releases the draw lock.
Else, it starts the script in step mode.

\end{funcdesc}

\begin{funcdesc}{fforward}{}
\end{funcdesc}

\begin{funcdesc}{delay}{i}
Set the draw delay in seconds.
\end{funcdesc}

\begin{funcdesc}{sleep}{timeout=None}
Sleep until key/mouse press in the canvas or until timeout
\end{funcdesc}

\begin{funcdesc}{wakeup}{mode=0}
Wake up from the sleep function.

This is the only way to exit the sleep() function.
Default is to wake up from the current sleep. A mode > 0
forces wakeup for longer period.

\end{funcdesc}

\begin{funcdesc}{exit}{all=False}
Exit from the current script or from pyformex if no script running.
\end{funcdesc}

\begin{funcdesc}{printbbox}{}
\end{funcdesc}

\begin{funcdesc}{printviewportsettings}{}
\end{funcdesc}

\begin{funcdesc}{layout}{nvps=None,ncols=None,nrows=None}
Set the viewports layout.
\end{funcdesc}

\begin{funcdesc}{addViewport}{}
Add a new viewport.
\end{funcdesc}

\begin{funcdesc}{removeViewport}{}
Remove a new viewport.
\end{funcdesc}

\begin{funcdesc}{linkViewport}{vp,tovp}
Link viewport vp to viewport tovp.

Both vp and tovp should be numbers of viewports. 

\end{funcdesc}

\begin{funcdesc}{viewport}{n}
Select the current viewport
\end{funcdesc}

\begin{funcdesc}{updateGUI}{}
Update the GUI.
\end{funcdesc}

\begin{funcdesc}{flyAlong}{path='flypath',upvector=[0.,1.,0.],sleeptime=None}
Fly through the scene along the flypath.
\end{funcdesc}

\begin{funcdesc}{highlightActors}{K,colormap=highlight_colormap}
Highlight a selection of actors on the canvas.

K is Collection of actors as returned by the pick() method.
colormap is a list of two colors, for the actors not in, resp. in
the Collection K.

\end{funcdesc}

\begin{funcdesc}{highlightElements}{K,colormap=highlight_colormap}
Highlight a selection of actor elements on the canvas.

K is Collection of actor elements as returned by the pick() method.
colormap is a list of two colors, for the elements not in, resp. in
the Collection K.

\end{funcdesc}

\begin{funcdesc}{highlightEdges}{K,colormap=highlight_colormap}
Highlight a selection of actor edges on the canvas.

K is Collection of TriSurface actor edges as returned by the pick() method.
colormap is a list of two colors, for the edges not in, resp. in
the Collection K.

\end{funcdesc}

\begin{funcdesc}{highlightPoints}{K,colormap=highlight_colormap}
Highlight a selection of actor elements on the canvas.

K is Collection of actor elements as returned by the pick() method.


\end{funcdesc}

\begin{funcdesc}{highlightPartitions}{K}
Highlight a selection of partitions on the canvas.

K is a Collection of actor elements, where each actor element is
connected to a collection of property numbers, as returned by the
partitionCollection() method.

\end{funcdesc}

\begin{funcdesc}{set_selection_filter}{i}
Set the selection filter mode
\end{funcdesc}

\begin{funcdesc}{pick}{mode='actor',single=False,func=None,filtr=None,numbers=False}
Enter interactive picking mode and return selection.

See viewport.py for more details.
This function differs in that it provides default highlighting
during the picking operation, a button to stop the selection operation

If no filter is given, the available filters are presented in a combobox.
If numbers=True, numbers are shown?? Sophie: can you explain some more
what it is doing?

\end{funcdesc}

\begin{funcdesc}{pickActors}{single=False,func=None,filtr=None}
\end{funcdesc}

\begin{funcdesc}{pickElements}{single=False,func=None,filtr=None}
\end{funcdesc}

\begin{funcdesc}{pickPoints}{single=False,func=None,filtr=None}
\end{funcdesc}

\begin{funcdesc}{pickEdges}{single=False,func=None,filtr=None}
\end{funcdesc}

\begin{funcdesc}{highlight}{K,mode,colormap=highlight_colormap}
Highlight a Collection of actor/elements.

K is usually the return value of a pick operation, but might also
be set by the user.
mode is one of the pick modes.

\end{funcdesc}

\begin{funcdesc}{pickNumbers}{marks=None}
\end{funcdesc}

\begin{funcdesc}{showNumbers}{}
Show a list widget containing the numbers of the elements,
points or edges. When part numbers are added to / removed from
the selection by mouse_picking, the corresponding list items will be
selected / deselected. Part numbers can also be added to / removed
from the selection by clicking on a list item.

\end{funcdesc}

\begin{funcdesc}{hideNumbers}{}
Hide the list widget.
\end{funcdesc}

\begin{funcdesc}{set_edit_mode}{i}
Set the drawing edit mode.
\end{funcdesc}

\begin{funcdesc}{drawLinesInter}{mode ='line',single=False,func=None}
Enter interactive drawing mode and return the line drawing.

See viewport.py for more details.
This function differs in that it provides default displaying
during the drawing operation and a button to stop the drawing operation.

The drawing can be edited using the methods 'undo', 'clear' and 'close', which
are presented in a combobox.

\end{funcdesc}

\begin{funcdesc}{showLineDrawing}{L}
Show a line drawing.

L is usually the return value of an interactive draw operation, but
might also be set by the user.

\end{funcdesc}

\begin{funcdesc}{setLocalAxes}{mode=True}
\end{funcdesc}

\begin{funcdesc}{setGlobalAxes}{mode=True}
\end{funcdesc}

\begin{funcdesc}{Export}{dict}
\end{funcdesc}


\subsection{Render mode}

The \pyformex drawing engine has multiple rendering modes. Depending on the task you are trying to perform, one of the rendering modes may be more suited that the others. While smooth opaque rendering with lighting enabled may produce the most realistic view of some final object, modes that allow some transparency through the model may be better suited at many stages of the building of the object.

The different rendering modes in \pyformex are:
\begin{description}
  \item[wireframe] All objects are displayed only with lines marking its edges. A line element is drawn as a line; a triangle element is drawn as three lines along its border. The wireframe mode permits transparency through the model and is computationally efficient. It is the default mode and mostly used during model construction.
  \item[smooth] Smooth rendering will try to display the objects in a very realisitc way. 
\end{description}


\begin{funcdesc}{renderMode}{mode}
Set the rendermode for the current viewport. The argument is the name of one of the defined rendering modes.
\end{funcdesc}

The following convenience functions provide an easy alternative for switching to the corresponding rendering mode.
    
\begin{funcdesc}{wireframe}{}
\end{funcdesc}
\begin{funcdesc}{flat}{}
\end{funcdesc}
\begin{funcdesc}{flatwire}{}
\end{funcdesc}
\begin{funcdesc}{smooth}{}
\end{funcdesc}
\begin{funcdesc}{smoothwire}{}
\end{funcdesc}


%%%%%%%%%%%%%%%%%%%%%%%%%%%%%%%%%%%%%%%%%%%%%%%%%%%%%%%%%%%%%%%%%%%%%%%%%%%%%

\section{Utilities}
\label{sec:utilities}

The \Code{pyformex.utils} module provides general functions and classes that are used in different places of \pyformex. Some of these functions may be very useful to the user and are described below.

\begin{classdesc}{NameSequence}{}
A class for autogenerating sequences of names.

The name includes a numeric part, whose number is incremented
at each call of the 'next()' method.

This class is often used for automatically generating filename families.
    
\begin{memberdesc}{__init__}{name,ext=''}
Create a new NameSequence from name,ext.

If the name starts with a non-numeric part, it is taken as a constant part.
If the name ends with a numeric part, the next generated names will
be obtained by incrementing this part.
If not, a string '-000' will be appended and names will be generated
by incrementing this part.

If an extension is given, it will be appended as is to the names.
This makes it possible to put the numeric part anywhere inside the
names.

Examples:
\Code{NameSequence('hallo.98')} will generate names hallo.98, hallo.99, hallo.100, ...
\Code{NameSequence('hallo','.png')} will generate names hallo-000.png, hallo-001.png, ...
\Code{NameSequence('/home/user/hallo23','5.png')} will generate names
/home/user/hallo235.png, /home/user/hallo245.png, ...
"""
\end{memberdesc}

\begin{memberdesc}{next}
Returns the next name in the sequence.
\end{memberdesc}

\begin{memberdesc}{peek}
Returns the next name in the sequence without actually incrementing the counter.
The next call to peek() or next() will return the same name.
\end{memberdesc}

\begin{memberdesc}{glob}
Returns a UNIX-style glob pattern for the generated names.

A NameSequence is often used as a generator for file names.
The glob() method returns a pattern that can be used in a
UNIX-like shell command to select all the generated file names.
\end{memberdesc}
\end{classdesc}


\begin{funcdesc}{splitEndDigits}{s}
    Split a string in any prefix and a numerical end sequence.

    A string like 'abc-0123' will be split in 'abc-' and '0123'.
    Any of both can be empty.
\end{funcdesc} 




%%% Local Variables: 
%%% mode: latex
%%% TeX-master: "pyformex"
%%% End: 
