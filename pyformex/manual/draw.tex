% pyformex manual --- draw
% $Id$
% (C) B.Verhegghe

\section{\module{draw} --- Create graphical representations}
\label{sec:drawing}

\declaremodule{""}{draw}
\modulesynopsis{Create graphical representations of 3D geometries.}
\moduleauthor{'pyFormex project'}{'http://pyformex.berlios.de'}

The following functions provide access to the OpenGL 3D drawing capabilities of \pyformex.

 
\begin{funcdesc}{draw}{F, view=None, bbox='auto', color='prop', colormap=None, wait=True, eltype=None, allviews=False, marksize=None, linewidth=None, alpha=0.5}

Draw a Formex or a list of Formices on the canvas.

    If F is a list, all its items are drawn with the same settings.

    If a setting is unspecified, its current values are used.
    
    Draws an actor on the canvas, and directs the camera to it from
    the specified view. Named views are either predefined or can be added by
    the user.
    If view=None is specified, the camera settings remain unchanged.
    This may make the drawn object out of view!
    A special name '__last__' may be used to keep the same camera angles
    as in the last draw operation. The camera will be zoomed on the newly
    drawn object.
    The initial default view is 'front' (looking in the -z direction).

    If bbox == 'auto', the camera will zoom automatically on the shown
    object. A bbox may be specified to have other zoom settings, e.g. to
    keep the previous settings. If bbox == None, the previous bbox will be
    kept.

    If other actors are on the scene, they may or may not be visible with the
    new camera settings. Clear the canvas before drawing if you only want
    a single actor!

    If the Formex has properties and a color list is specified, then the
    the properties will be used as an index in the color list and each member
    will be drawn with the resulting color.
    If color is one color value, the whole Formex will be drawn with
    that color.
    Finally, if color=None is specified, the whole Formex is drawn in black.
    
    Each draw action activates a locking mechanism for the next draw action,
    which will only be allowed after drawdelay seconds have elapsed. This
    makes it easier to see subsequent images and is far more elegant that an
    explicit sleep() operation, because all script processing will continue
    up to the next drawing instruction.
    The value of drawdelay is set in the config, or 2 seconds by default.
    The user can disable the wait cycle for the next draw operation by
    specifying wait=False. Setting drawdelay=0 will disable the waiting
    mechanism for all subsequent draw statements (until set >0 again).

\end{funcdesc}

\subsection{Render mode}

The \pyformex drawing engine has multiple rendering modes. Depending on the task you are trying to perform, one of the rendering modes may be more suited that the others. While smooth opaque rendering with lighting enabled may produce the most realistic view of some final object, modes that allow some transparency through the model may be better suited at many stages of the building of the object.

The different rendering modes in \pyformex are:
\begin{description}
  \item[wireframe] All objects are displayed only with lines marking its edges. A line element is drawn as a line; a triangle element is drawn as three lines along its border. The wireframe mode permits transparency through the model and is computationally efficient. It is the default mode and mostly used during model construction.
  \item[smooth] Smooth rendering will try to display the objects in a very realisitc way. 
\end{description}


\begin{funcdesc}{renderMode}{mode}
Set the rendermode for the current viewport. The argument is the name of one of the defined rendering modes.
\end{funcdesc}

The following convenience functions provide an easy alternative for switching to the corresponding rendering mode.
    
\begin{funcdesc}{wireframe}{}
\end{funcdesc}
\begin{funcdesc}{flat}{}
\end{funcdesc}
\begin{funcdesc}{flatwire}{}
\end{funcdesc}
\begin{funcdesc}{smooth}{}
\end{funcdesc}
\begin{funcdesc}{smoothwire}{}
\end{funcdesc}


%%%%%%%%%%%%%%%%%%%%%%%%%%%%%%%%%%%%%%%%%%%%%%%%%%%%%%%%%%%%%%%%%%%%%%%%%%%%%

\section{Utilities}
\label{sec:utilities}

The \Code{pyformex.utils} module provides general functions and classes that are used in different places of \pyformex. Some of these functions may be very useful to the user and are described below.

\begin{classdesc}{NameSequence}{}
A class for autogenerating sequences of names.

The name includes a numeric part, whose number is incremented
at each call of the 'next()' method.

This class is often used for automatically generating filename families.
    
\begin{memberdesc}{__init__}{name,ext=''}
Create a new NameSequence from name,ext.

If the name starts with a non-numeric part, it is taken as a constant part.
If the name ends with a numeric part, the next generated names will
be obtained by incrementing this part.
If not, a string '-000' will be appended and names will be generated
by incrementing this part.

If an extension is given, it will be appended as is to the names.
This makes it possible to put the numeric part anywhere inside the
names.

Examples:
\Code{NameSequence('hallo.98')} will generate names hallo.98, hallo.99, hallo.100, ...
\Code{NameSequence('hallo','.png')} will generate names hallo-000.png, hallo-001.png, ...
\Code{NameSequence('/home/user/hallo23','5.png')} will generate names
/home/user/hallo235.png, /home/user/hallo245.png, ...
"""
\end{memberdesc}

\begin{memberdesc}{next}
Returns the next name in the sequence.
\end{memberdesc}

\begin{memberdesc}{peek}
Returns the next name in the sequence without actually incrementing the counter.
The next call to peek() or next() will return the same name.
\end{memberdesc}

\begin{memberdesc}{glob}
Returns a UNIX-style glob pattern for the generated names.

A NameSequence is often used as a generator for file names.
The glob() method returns a pattern that can be used in a
UNIX-like shell command to select all the generated file names.
\end{memberdesc}
\end{classdesc}


\begin{funcdesc}{splitEndDigits}{s}
    Split a string in any prefix and a numerical end sequence.

    A string like 'abc-0123' will be split in 'abc-' and '0123'.
    Any of both can be empty.
\end{funcdesc} 




%%% Local Variables: 
%%% mode: latex
%%% TeX-master: "pyformex"
%%% End: 
