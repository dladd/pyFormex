% pyformex manual --- script
% $Id$
% (C) B.Verhegghe

\section{\module{script} --- Basic pyFormex script functions}
\label{sec:script}

\declaremodule{""}{script}
\modulesynopsis{Basic pyFormex script functions}
\moduleauthor{'pyFormex project'}{'http://pyformex.berlios.de'}

The \Code{pyformex.script} module provides the basic functions available
in all \pyformex scripts. These functions are available in GUI and NONGUI
applications, without the need to explicitely importing the \module{script}
module.



\begin{funcdesc}{rename}{}
Rename the global variables in oldnames to newnames.

\end{funcdesc}


\begin{funcdesc}{named}{}
Returns the global object named name.

\end{funcdesc}


\begin{funcdesc}{export2}{}
Export a list of names and values.

\end{funcdesc}


\begin{funcdesc}{forget}{}
Remove the global variables specified in list.

\end{funcdesc}


\begin{funcdesc}{formatInfo}{}
Return formatted information about a Formex.

\end{funcdesc}


\begin{funcdesc}{workHere}{}
Change the current working directory to the script's location.

\end{funcdesc}


\begin{funcdesc}{force_finish}{}


\end{funcdesc}


\begin{funcdesc}{warning}{}


\end{funcdesc}


\begin{funcdesc}{export}{}
Export the variables in the given dictionary.

\end{funcdesc}


\begin{funcdesc}{runtime}{}
Return the time elapsed since start of execution of the script.

\end{funcdesc}


\begin{funcdesc}{stopatbreakpt}{}
Set the exitrequested flag.

\end{funcdesc}


\begin{funcdesc}{printconfig}{}


\end{funcdesc}


\begin{funcdesc}{system}{}
Run a command and return its output.

    If result == 'status', the exit status of the command is returned.
    If result == 'output', the output of the command is returned.
    If result == 'both', a tuple of status and output is returned.
    

\end{funcdesc}


\begin{funcdesc}{exit}{}
Exit from the current script or from pyformex if no script running.

\end{funcdesc}


\begin{funcdesc}{playScript}{}
Play a pyformex script scr. scr should be a valid Python text.

    There is a lock to prevent multiple scripts from being executed at the
    same time. This implies that pyFormex scripts can currently not be
    recurrent.
    If a name is specified, set the global variable GD.scriptName to it
    when the script is started.
    If a filename is specified, set the global variable __file__ to it.
    
    If step==True, an indefinite pause will be started after each line of
    the script that starts with 'draw'. Also (in this case), each line
    (including comments) is echoed to the message board.
    

\end{funcdesc}


\begin{funcdesc}{play}{}
Play a formex script from file fn or from the current file.

    This function does nothing if no file is passed or no current
    file was set.
    

\end{funcdesc}


\begin{funcdesc}{runApp}{}
Run the application without gui.

\end{funcdesc}


\begin{funcdesc}{printglobals}{}


\end{funcdesc}


\begin{funcdesc}{playFile}{}
Play a formex script from file fn.

    fn is the name of a file holding a pyFormex script.
    A list of arguments can be passed. They will be available under the name
    argv. This variable can be changed by the script and the resulting argv
    is returned to the caller.
    

\end{funcdesc}


\begin{funcdesc}{ask}{}
Ask a question and present possible answers.

    If no choices are presented, anything will be accepted.
    Else, the question is repeated until one of the choices is selected.
    If a default is given and the value entered is empty, the default is
    substituted.
    Case is not significant, but choices are presented unchanged.
    If no choices are presented, the string typed by the user is returned.
    Else the return value is the lowest matching index of the users answer
    in the choices list. Thus, ask('Do you agree',['Y','n']) will return
    0 on either 'y' or 'Y' and 1 on either 'n' or 'N'.
    

\end{funcdesc}


\begin{funcdesc}{breakpt}{}
Set a breakpoint where the script can be halted on a signal.

    If an argument is specified, it will be written to the message board.

    The exitrequested signal is usually emitted by pressing a button in the GUI.
    In nongui mode the stopatbreakpt function can be called from another thread.
    

\end{funcdesc}


\begin{funcdesc}{showInfo}{}


\end{funcdesc}


\begin{funcdesc}{printglobalnames}{}


\end{funcdesc}


\begin{funcdesc}{ack}{}
Show a Yes/No question and return True/False depending on answer.

\end{funcdesc}


\begin{funcdesc}{chdir}{}
Change the current working directory.

    If fn is a directory name, the current directory is set to fn.
    If fn is a file name, the current directory is set to the directory
    holding fn.
    In either case, the current dirctory is stored in GD.cfg['workdir']
    for persistence between pyFormex invocations.
    
    If fn does not exist, nothing is done.
    

\end{funcdesc}


\begin{funcdesc}{enableBreak}{}


\end{funcdesc}


\begin{funcdesc}{printall}{}
Print all Formices in globals()

\end{funcdesc}


\begin{funcdesc}{Globals}{}
Return the globals that are passed to the scripts on execution.

    This basically contains the globals defined in draw.py, colors.py,
    and formex.py, as well as the globals from numpy.
    It also contains the definitions put into the pyformex.PF, by
    preference using the export() function.
    During execution of the script, the global variable __name__ will be
    set to either 'draw' or 'script' depending on whether the script
    was executed in the 'draw' module (--gui option) or the 'script'
    module (--nogui option).
    

\end{funcdesc}


\begin{funcdesc}{error}{}
Show an error message and wait for user acknowlegement.

\end{funcdesc}


\begin{funcdesc}{step_script}{}


\end{funcdesc}


\begin{funcdesc}{listAll}{}
Return a list of all objects in dic that are of given clas.

    If no class is given, Formex objects are sought.
    If no dict is given, the objects from both GD.PF and locals()
    are returned.
    

\end{funcdesc}


%%% Local Variables: 
%%% mode: latex
%%% TeX-master: "pyformex"
%%% End:

