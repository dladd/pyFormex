% pyformex manual --- odict
% $Id$
% (C) Benedict Verhegghe (benedict.verhegghe@ugent.be)
% DO NOT EDIT THIS FILE: it was automatically generated by 'gendoc.py'


\section{\module{odict} --- A dictionary that keeps the keys in order of insertion.}
\label{sec:odict}

\declaremodule{extension}{pyformex.odict}
\modulesynopsis{A dictionary that keeps the keys in order of insertion.}
\moduleauthor{'pyFormex project'}{'http://pyformex.org'}



\subsection{ODict --- An ordered dictionary.}
    This is a dictionary that keeps the keys in order.
    The default order is the insertion order. The current order can be
    changed at any time.
    

\begin{classdesc}{ODict}{data={}}
Create a new ODict instance.

        The ODict can be initialized with a Python dict or an ODict.
        The order after insertion is indeterminate if a plain dict is used.
        
\end{classdesc}

ODict instances have the following methods:

\begin{methoddesc}{__repr__}{}
Format the Dict as a string.

        We use the format Dict({}), so that the string is a valid Python
        representation of the Dict.
        
\end{methoddesc}

\begin{methoddesc}{__setitem__}{key,value}
Allows items to be set using self[key] = value.
\end{methoddesc}

\begin{methoddesc}{__delitem__}{key}
Allow items to be deleted using del self[key].

        Raises an error if key does not exist.
        
\end{methoddesc}

\begin{methoddesc}{update}{data={}}
Add a dictionary to the ODict object.

        The new keys will be appended to the existing, but the order of the
        added keys is undetemined if data is a dict object. If data is an ODict
        its order will be respected.. 
        
\end{methoddesc}

\begin{methoddesc}{__add__}{data}
Add two ODicts's together, returning the result.
\end{methoddesc}

\begin{methoddesc}{sort}{keys}
Set the order of the keys.

        keys should be a list containing exactly all the keys from self.
        
\end{methoddesc}

\begin{methoddesc}{keys}{}
Return the keys in order.
\end{methoddesc}

\begin{methoddesc}{values}{}
Return the values in order of the keys.
\end{methoddesc}

\begin{methoddesc}{items}{}
Return the key,value pairs in order of the keys.
\end{methoddesc}

\subsection{KeyList --- A named item list.}
    A KeyList is a list of lists or tuples. Each item (sublist or tuple)
    should at least have 2 elements: the first one is used as a key to
    identify the item, but is also part of the information (value) of the
    item.
    

\begin{classdesc}{KeyList}{alist=[]}
Create a new KeyList, possibly filling it with data.

        data should be a list of tuples/lists each having at
        least 2 elements.
        The (string value of the) first is used as the key.
        
\end{classdesc}

KeyList instances have the following methods:

\begin{methoddesc}{items}{}
Return the key+value lists in order of the keys.
\end{methoddesc}

\subsection{Functions defined in module \module{odict}}

\begin{funcdesc}{listUnion}{a,b}
This function is deprecated.
\end{funcdesc}

\begin{funcdesc}{listDifference}{a,b}
This function is deprecated.
\end{funcdesc}

\begin{funcdesc}{listSymDifference}{a,b}
This function is deprecated.
\end{funcdesc}

\begin{funcdesc}{listIntersection}{a,b}
This function is deprecated.
\end{funcdesc}

\begin{funcdesc}{listSelect}{a,b}
This function is deprecated.
\end{funcdesc}

\begin{funcdesc}{listConcatenate}{a}
This function is deprecated.
\end{funcdesc}

\begin{funcdesc}{listFlatten}{a}
This function is deprecated.
\end{funcdesc}

%%% Local Variables: 
%%% mode: latex
%%% TeX-master: "pyformex"
%%% End:

