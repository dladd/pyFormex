% pyformex manual --- colorscale --- CREATED WITH py2ptex.py: DO NOT EDIT
% $Id$
% (C) B.Verhegghe

\section{\module{colorscale} --- Color mapping of a range of values.}
\label{sec:colorscale}

\declaremodule{""}{colorscale}
\modulesynopsis{Color mapping of a range of values.}
\moduleauthor{'pyFormex project'}{'http://pyformex.berlios.de'}




\subsection{ColorScale class: Mapping floating point values into colors.}

    A colorscale maps floating point values within a certain range
    into colors and can be used to provide visual representation
    of numerical values.
    This is e.g. quite useful in Finite Element postprocessing (see the
    postproc plugin).

    The ColorLegend class provides a way to make the ColorScale visible
    on the canvas.
    


The ColorScale class has this constructor: 

\begin{classdesc}{ColorScale}{palet,minval=0.,maxval=1.,midval=None,exp=1.0,exp2=None}
Create a colorscale to map a range of values into colors.

        The values range from minval to maxval (default 0.0..1.0).

        A midval may be specified to set the value corresponding to
        the midle of the color scale. It defaults to the middle value
        of the range. It is especially useful if the range extends over
        negative and positive values to set 0.0 as the middle value. 

        The palet is a list of 3 colors, corresponding to the minval,
        midval and maxval respectively. The middle color may be given
        as None, in which case it will be set to the middle color
        between the first and last.

        The Palette variable provides some useful predefined palets.
        You will hardly ever need to define your own palets.

        The mapping function between numerical and color values is by
        default linear. Nonlinear mappings can be obtained by specifying
        an exponent 'exp' different from 1.0. Mapping is done with the
        'stuur' function from the 'utils' module. 
        If 2 exponents are given, mapping is done independently e with exp
        in the range minval..midval and with exp2 in the range midval..maxval.
        

\end{classdesc}

ColorScale objects have the following methods:

\begin{funcdesc}{scale}{val}
Scale a value to the range -1...1.

        If the ColorScale has only one exponent, values in the range
        mival..maxval are scaled to the range -1..+1.

        If two exponents wer specified, scaling is done independently in
        one of the intervals minval..midval or midval..maxval resulting into
        resp. the interval -1..0 or 0..1.
        
\end{funcdesc}

\begin{funcdesc}{color}{val}
Return the color representing a value val.

        The returned color is a tuple of three RGB values in the range 0-1.
        The color is obtained by first scaling the value to the -1..1 range
        using the 'scale' method, and then using that result to pick a color
        value from the palet. A palet specifies the three colors corresponding
        to the -1, 0 and 1 values.
        
\end{funcdesc}


\subsection{ColorLegend class: A colorlegend is a colorscale divided in a number of subranges.}




The ColorLegend class has this constructor: 

\begin{classdesc}{ColorLegend}{colorscale,n}
Create a color legend dividing a colorscale in n subranges.

        The full value range of the colorscale is divided in n subranges,
        each half range being divided in n/2 subranges.
        This sets n+1 limits of the subranges.
        The n colors of the subranges correspond to the subrange middle value.
        

\end{classdesc}

ColorLegend objects have the following methods:

\begin{funcdesc}{overflow}{oflow=None}
Raise a runtime error if oflow == None, else return oflow.
\end{funcdesc}

\begin{funcdesc}{color}{val}
Return the color representing a value val.

        The color is that of the subrange holding the value. If the value
        matches a subrange limit, the lower range color is returned.
        If the value falls outside the colorscale range, a runtime error
        is raised, unless the corresponding underflowcolor or overflowcolor
        attribute has been set, in which case this attirbute is returned.
        Though these attributes can be set to any not None value, it will
        usually be set to some color value, that will be used to show
        overflow values.
        The returned color is a tuple of three RGB values in the range 0-1.
        
\end{funcdesc}


%%% Local Variables: 
%%% mode: latex
%%% TeX-master: "pyformex"
%%% End:

