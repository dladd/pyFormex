% pyformex manual --- lima
% $Id$
% (C) Benedict Verhegghe (benedict.verhegghe@ugent.be)
% DO NOT EDIT THIS FILE: it was automatically generated by 'gendoc.py'


\section{\module{lima} --- Lindenmayer Systems}
\label{sec:lima}

\declaremodule{extension}{pyformex.plugins.lima}
\modulesynopsis{Lindenmayer Systems}
\moduleauthor{'pyFormex project'}{'http://pyformex.org'}



\subsection{Lima --- A class for operations on Lindenmayer Systems.}


\begin{classdesc}{Lima}{axiom="",rules={}}

\end{classdesc}

Lima instances have the following methods:

\begin{methoddesc}{status}{}
Print the status of the Lima
\end{methoddesc}

\begin{methoddesc}{addRule}{atom,product}
Add a new rule (or overwrite an existing)
\end{methoddesc}

\begin{methoddesc}{translate}{rule,keep=False}
Translate the product by the specified rule set.

        If keep=True is specified, atoms that do not have a translation
        in the rule set, will be kept unchanged.
        The default (keep=False) is to remove those atoms.
        
\end{methoddesc}

\begin{methoddesc}{grow}{ngen=1}

\end{methoddesc}

\subsection{Functions defined in module \module{lima}}

\begin{funcdesc}{lima}{axiom,rules,level,turtlecmds,glob=None}
Create a list of connected points using a Lindenmayer system.

    axiom is the initial string,
    rules are translation rules for the characters in the string,
    level is the number of generations to produce,
    turtlecmds are the translation rules of the final string to turtle cmds,
    glob is an optional list of globals to pass to the turtle script player.

    This is a convenience function for quickly creating a drawing of a
    single generation member. If you intend to draw multiple generations
    of the same Lima, it is better to use the grow() and translate() methods
    directly.
    
\end{funcdesc}

%%% Local Variables: 
%%% mode: latex
%%% TeX-master: "pyformex"
%%% End:

