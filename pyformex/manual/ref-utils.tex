% pyformex manual --- utils
% $Id$
% (C) B.Verhegghe

\section{\module{utils} --- General purpose utilities}
\label{sec:utils}

\declaremodule{""}{utils}
\modulesynopsis{General utitlity functions for pyFormex.}
\moduleauthor{'pyFormex project'}{'http://pyformex.berlios.de'}


The \Code{pyformex.utils} module provides general functions and classes that are used in different places of \pyformex. Some of these functions may be very useful to the user and are described below.


%%%%%%%%%%%%%%%%%%%%%%%%%%%%%%%%%%%%%%%%%%%%%%%%%%%%%%%
\begin{classdesc}{NameSequence}{}
A class for autogenerating sequences of names.

The name includes a numeric part, whose number is incremented
at each call of the 'next()' method.

This class is often used for automatically generating filename families.
    
\begin{memberdesc}{__init__}{name,ext=''}
Create a new NameSequence from name,ext.

If the name starts with a non-numeric part, it is taken as a constant part.
If the name ends with a numeric part, the next generated names will
be obtained by incrementing this part.
If not, a string '-000' will be appended and names will be generated
by incrementing this part.

If an extension is given, it will be appended as is to the names.
This makes it possible to put the numeric part anywhere inside the
names.

Examples:
\Code{NameSequence('hallo.98')} will generate names hallo.98, hallo.99, hallo.100, ...
\Code{NameSequence('hallo','.png')} will generate names hallo-000.png, hallo-001.png, ...
\Code{NameSequence('/home/user/hallo23','5.png')} will generate names
/home/user/hallo235.png, /home/user/hallo245.png, ...
"""
\end{memberdesc}

\begin{memberdesc}{next}
Returns the next name in the sequence.
\end{memberdesc}

\begin{memberdesc}{peek}
Returns the next name in the sequence without actually incrementing the counter.
The next call to peek() or next() will return the same name.
\end{memberdesc}

\begin{memberdesc}{glob}
Returns a UNIX-style glob pattern for the generated names.

A NameSequence is often used as a generator for file names.
The glob() method returns a pattern that can be used in a
UNIX-like shell command to select all the generated file names.
\end{memberdesc}
\end{classdesc}


\begin{funcdesc}{splitEndDigits}{s}
    Split a string in any prefix and a numerical end sequence.

    A string like 'abc-0123' will be split in 'abc-' and '0123'.
    Any of both can be empty.
\end{funcdesc} 




%%% Local Variables: 
%%% mode: latex
%%% TeX-master: "pyformex"
%%% End: 
