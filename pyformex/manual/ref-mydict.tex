% pyformex manual --- mydict
% $Id$
% (C) Benedict Verhegghe (benedict.verhegghe@ugent.be)
% DO NOT EDIT THIS FILE: it was automatically generated by 'gendoc.py'


\section{\module{mydict} --- }
\label{sec:mydict}

\declaremodule{extension}{pyformex.mydict}
\modulesynopsis{}
\moduleauthor{'pyFormex project'}{'http://pyformex.org'}

Dict is a dictionary with default values and alternate attribute syntax.
CDict is a Dict with lookup cascading into the next level Dict's
if the key is not found in the CDict itself.

(C) 2005,2008 Benedict Verhegghe
Distributed under the GNU GPL version 3 or later


\subsection{Dict --- A Python dictionary with default values and attribute syntax.}
    Dict is functionally nearly equivalent with the builtin Python dict,
    but provides the following extras:
    - Items can be accessed with attribute syntax as well as dictionary
      syntax. Thus, if C is a Dict, the following are equivalent:
          C['foo']   or   C.foo
      This works as well for accessing values as for setting values.
      In the following, the words key or attribute therefore have the
      same meaning.
    - Lookup of a nonexisting key/attribute does not automatically raise an
      error, but calls a _default_ lookup method which can be set by the user.
      The default is to raise a KeyError, but an alternative is to return
      None or some other default value.

    There are a few caveats though:
    - Keys that are also attributes of the builtin dict type, can not be used
      with the attribute syntax to get values from the Dict. You should use
      the dictionary syntax to access these items. It is possible to set
      such keys as attributes. Thus the following will work:
         C['get'] = 'foo'
         C.get = 'foo'
         print C['get']
      but not
         print C.get

      This is done so because we want all the dict attributes to be available
      with their normal binding. Thus,
         print C.get('get')
      will print
         foo

    To avoid name clashes with user defines, many Python internal names start
    and end with '__'. The user should avoid such names.
    The Python dict has the following attributes not enclosed between '__',
    so these are the one to watch out for:
    'clear', 'copy', 'fromkeys', 'get', 'has_key', 'items', 'iteritems',
    'iterkeys', 'itervalues', 'keys', 'pop', 'popitem', 'setdefault',
    'update', 'values'.
    

\begin{classdesc}{Dict}{data={},default=None}
Create a new Dict instance.

        The Dict can be initialized with a Python dict or a Dict.
        If defined, default is a function that is used for alternate key
        lookup if the key was not found in the dict.
        
\end{classdesc}

Dict instances have the following methods:

\begin{methoddesc}{__repr__}{}
Format the Dict as a string.

        We use the format Dict({}), so that the string is a valid Python
        representation of the Dict.
        
\end{methoddesc}

\begin{methoddesc}{__getitem__}{key}
Allows items to be addressed as self[key].

        This is equivalent to the dict lookup, except that we
        provide a default value if the key does not exist.
        
\end{methoddesc}

\begin{methoddesc}{__delitem__}{key}
Allow items to be deleted using del self[key].

        Silently ignore if key is nonexistant.
        
\end{methoddesc}

\begin{methoddesc}{__getattr__}{key}
Allows items to be addressed as self.key.

        This makes self.key equivalent to self['key'], except if key
        is an attribute of the builtin type 'dict': then we return that
        attribute instead, so that the 'dict' methods keep their binding.
        
\end{methoddesc}

\begin{methoddesc}{__setattr__}{key,value=None}
Allows items to be set as self.key=value.

        This works even if the key is an existing attribute of the
        builtin dict class: the key,value pair is stored in the dict,
        leaving the dict's attributes unchanged.
        
\end{methoddesc}

\begin{methoddesc}{__delattr__}{key}
Allow items to be deleted using del self.key.

        This works even if the key is an existing attribute of the
        builtin dict class: the item is deleted from the dict,
        leaving the dict's attributes unchanged.
        
\end{methoddesc}

\begin{methoddesc}{update}{data={}}
Add a dictionary to the Dict object.

        The data can be a dict or Dict type object. 
        
\end{methoddesc}

\begin{methoddesc}{get}{key, default}
Return the value for key or a default.

        This is the equivalent of the dict get method, except that it
        returns only the default value if the key was not found in self,
        and there is no _default_ method or it raised a KeyError.
        
\end{methoddesc}

\begin{methoddesc}{setdefault}{key, default}
Replaces the setdefault function of a normal dictionary.

        This is the same as the get method, except that it also sets the
        default value if get found a KeyError.
        
\end{methoddesc}

\begin{methoddesc}{__deepcopy__}{memo}
Create a deep copy of ourself.
\end{methoddesc}

\begin{methoddesc}{__reduce__}{}

\end{methoddesc}

\begin{methoddesc}{__setstate__}{state}

\end{methoddesc}

\subsection{CDict --- A cascading Dict: properties not in Dict are searched in all Dicts.}
    This is equivalent to the Dict class, except that if a key is not found
    and the CDict has items with values that are themselves instances
    of Dict or CDict, the key will be looked up in those Dicts as well.

    As you expect, this will make the lookup cascade into all lower levels
    of CDict's. The cascade will stop if you use a Dict.
    There is no way to guarantee in which order the (Cascading)Dict's are
    visited, so if multiple Dicts on the same level hold the same key,
    you should know yourself what you are doing.
    

\begin{classdesc}{CDict}{data={},default=returnNone}

\end{classdesc}

CDict instances have the following methods:

\begin{methoddesc}{__repr__}{}
Format the CDict as a string.

        We use the format Dict({}), so that the string is a valid Python
        representation of the Dict.
        
\end{methoddesc}

\begin{methoddesc}{__str__}{}
Format a CDict into a string.
\end{methoddesc}

\begin{methoddesc}{__getitem__}{key}
Allows items to be addressed as self[key].

        This is equivalent to the dict lookup, except that we
        cascade through lower level dict's.
        
\end{methoddesc}

\subsection{Functions defined in module \module{mydict}}

\begin{funcdesc}{cascade}{d, key}
Cascading lookup in a dictionary.

    This is equivalent to the dict lookup, except that when the key is
    not found, a cascading lookup through lower level dict's is started
    and the first matching key found is returned.
    
\end{funcdesc}

\begin{funcdesc}{returnNone}{key}
Always returns None.
\end{funcdesc}

\begin{funcdesc}{raiseKeyError}{key}
Raise a KeyError.
\end{funcdesc}

\begin{funcdesc}{__newobj__}{cls, *args}

\end{funcdesc}

%%% Local Variables: 
%%% mode: latex
%%% TeX-master: "pyformex"
%%% End:

