% pyformex manual --- curve --- CREATED WITH py2ptex.py: DO NOT EDIT
% $Id$
% (C) B.Verhegghe

\section{\module{curve} --- Definition of curves in pyFormex.}
\label{sec:curve}

\declaremodule{""}{curve}
\modulesynopsis{Definition of curves in pyFormex.}
\moduleauthor{'pyFormex project'}{'http://pyformex.berlios.de'}

(C) 2008 Benedict Verhegghe (bverheg at users.berlios.de)
I wrote this software in my free time, for my joy, not as a commissioned task.
Any copyright claims made by my employer should therefore be considered void.
Acknowledgements: Gianluca De Santis

This module defines classes and functions specialized for handling
one-dimensional geometry in pyFormex. These may be straight lines, polylines,
higher order curves and collections thereof. In general, the curves are 3D,
but special cases may be created for handling plane curves.



\subsection{Curve class: Base class for curve type classes.}

    This is a virtual class intended to be subclassed.
    It defines the common definitions for all curve types.
    The subclasses should at least define the following:
      sub_points(t,j)
    

Curve objects have the following methods:

\begin{funcdesc}{sub_points}{t,j}
Return the points at values t in part j

        t can be an array of parameter values, j is a single segment number.
        
\end{funcdesc}

\begin{funcdesc}{sub_points_2}{t,j}
Return the points at values,parts given by zip(t,j)

        t and j can both be arrays, but should have the same length.
        
\end{funcdesc}

\begin{funcdesc}{lengths}{}

\end{funcdesc}

\begin{funcdesc}{pointsAt}{t}
Returns the points at parameter values t.

        Parameter values are floating point values. Their integer part
        is interpreted as the curve segment number, and the decimal part
        goes from 0 to 1 over the segment.
        
\end{funcdesc}

\begin{funcdesc}{subPoints}{div=10,extend=[0.,0.]}
Return a series of points on the PolyLine.

        The parameter space of each segment is divided into ndiv parts.
        The coordinates of the points at these parameter values are returned
        as a Coords object.
        The extend parameter allows to extend the curve beyond the endpoints.
        The normal parameter space of each part is [0.0 .. 1.0]. The extend
        parameter will add a curve with parameter space [-extend[0] .. 0.0]
        for the first part, and a curve with parameter space
        [1.0 .. 1 + extend[0]] for the last part.
        The parameter step in the extensions will be adjusted slightly so
        that the specified extension is a multiple of the step size.
        If the curve is closed, the extend parameter is disregarded. 
        
\end{funcdesc}

\begin{funcdesc}{length}{}
Return the total length of the curve.

        This is only available for curves that implement the 'lengths'
        method.
        
\end{funcdesc}


\subsection{PolyLine class: A class representing a series of straight line segments.}




The PolyLine class has this constructor: 

\begin{classdesc}{PolyLine}{coords=[],closed=False}
Initialize a PolyLine from a coordinate array.

        coords is a (npts,3) shaped array of coordinates of the subsequent
        vertices of the polyline (or a compatible data object).
        If closed == True, the polyline is closed by connecting the last
        point to the first. This does not change the vertex data.
        

\end{classdesc}

PolyLine objects have the following methods:

\begin{funcdesc}{toFormex}{}
Return the polyline as a Formex.
\end{funcdesc}

\begin{funcdesc}{sub_points}{t,j}
Return the points at values t in part j
\end{funcdesc}

\begin{funcdesc}{sub_points2}{t,j}
Return the points at value,part pairs (t,j)
\end{funcdesc}

\begin{funcdesc}{vectors}{}
Return the vectors of the points to the next one.

        The vectors are returned as a Coords object.
        If not closed, this returns one less vectors than the number of points.
        
\end{funcdesc}

\begin{funcdesc}{directions}{}
Returns unit vectors in the direction of the next point.
\end{funcdesc}

\begin{funcdesc}{avgDirections}{normalized=True}
Returns average directions at the inner nodes.

        If open, the number of directions returned is 2 less than the
        number of points.
        
\end{funcdesc}

\begin{funcdesc}{lengths}{}
Return the length of the parts of the curve.
\end{funcdesc}

\begin{funcdesc}{atLength}{div}
Returns the parameter values for relative curve lengths div.
        
        'div' is a list of relative curve lengths (from 0.0 to 1.0).
        As a convenience, an single integer value may be specified,
        in which case the relative curve lengths are found by dividing
        the interval [0.0,1.0] in the specified number of subintervals.

        The function returns a list with the parameter values for the points
        at the specified relative lengths.
        
\end{funcdesc}

\begin{funcdesc}{reverse}{}

\end{funcdesc}


\subsection{Polygon class: A Polygon is a closed PolyLine.}




The Polygon class has this constructor: 

\begin{classdesc}{Polygon}{coords=[]}


\end{classdesc}

Polygon objects have the following methods:


\subsection{BezierSpline class: A class representing a Bezier spline curve.}




The BezierSpline class has this constructor: 

\begin{classdesc}{BezierSpline}{pts,deriv=None,curl=0.5,control=None,closed=False}
Create a cubic spline curve through the given points.

        The curve is defined by the points and the directions at these points.
        If no directions are specified, the average of the segments ending
        in that point is used, and in the end points of an open curve, the
        direction of the end segment.
        The curl parameter can be set to influence the curliness of the curve.
        curl=0.0 results in straight segment.
        

\end{classdesc}

BezierSpline objects have the following methods:

\begin{funcdesc}{sub_points}{t,j}

\end{funcdesc}


\subsection{CardinalSpline class: A class representing a cardinal spline.}




The CardinalSpline class has this constructor: 

\begin{classdesc}{CardinalSpline}{pts,tension=0.0,closed=False}
Create a natural spline through the given points.

        The Cardinal Spline with given tension is a Bezier Spline with curl:
            curl = ( 1 - tension) / 3
        The separate class name is retained for compatibility and convenience. 
        See CardinalSpline2 for a direct implementation (it misses the end
        intervals of the point set).
        

\end{classdesc}

CardinalSpline objects have the following methods:


\subsection{CardinalSpline2 class: A class representing a cardinal spline.}




The CardinalSpline2 class has this constructor: 

\begin{classdesc}{CardinalSpline2}{pts,tension=0.0,closed=False}
Create a natural spline through the given points.

        This is a direct implementation of the Cardinal Spline.
        For open curves, it misses the interpolation in the two end
        intervals of the point set.
        It is retained here because the implementation may some day
        replace the BezierSpline implementation.
        

\end{classdesc}

CardinalSpline2 objects have the following methods:

\begin{funcdesc}{compute_coefficients}{}

\end{funcdesc}

\begin{funcdesc}{sub_points}{t,j}

\end{funcdesc}


\subsection{NaturalSpline class: A class representing a natural spline.}




The NaturalSpline class has this constructor: 

\begin{classdesc}{NaturalSpline}{pts,endcond=['notaknot','notaknot'],closed=False}
Create a natural spline through the given points.

        pts specifies the coordinates of a set of points. A natural spline
        is constructed through this points.
        endcond specifies the end conditions in the first, resp. last point.
        It can be 'notaknot' or 'secder'.
        With 'notaknot', maximal continuity (up to the third derivative)
        is obtained between the first two splines.
        With 'secder', the spline ends with a zero second derivative.
        If closed is True, the spline is closed, and endcond is ignored.
        

\end{classdesc}

NaturalSpline objects have the following methods:

\begin{funcdesc}{compute_coefficients}{}

\end{funcdesc}

\begin{funcdesc}{sub_points}{t,j}

\end{funcdesc}


%%% Local Variables: 
%%% mode: latex
%%% TeX-master: "pyformex"
%%% End:

