% pyformex manual --- coords
% $Id$
% (C) B.Verhegghe


\section{\module{coords} --- Structured collections of 3D coordinates}
\label{sec:coords}

\declaremodule{}{coords}
\modulesynopsis{Storing and operating on large structured collections of 3D coordinates.}
\moduleauthor{'pyFormex project'}{'http://pyformex.berlios.de'}


The \module{coords} module implements a data class for storing large sets of 3D coordinates and provides a extensive set of methods for transforming these coordinates. The \class{Coords} class is used by other classes, such as \class{Formex} and \class{Surface}, which thus inherit the same transformation capabilities. In future, other geometrical data models may (and should) also derive from the \class{Coords} class. While the user will mostly use the higher level classes, he might occasionally find good reason to use the \class{Coords} class directly as well.

\subsection{The Coords class}
This class represents a structured set of 3D coordinates.
As the \class{Coords} class was written purposely to implement the \class{Formex} class, most of its features are exactly mirrored in the \class{Formex} class, and we refer to its description in \ref{sec:formex} for full details of all methods. Here we will merely point out the difference between a general instance of the \class{Coords} class and the \var{f} attribute of a \class{Formex} instance. 
The \class{Formex} class is always a 2-dimensional collection of \Code{(nelems*nplex)} points, each with 3 coordinates. A \class{Coords} object however is a collection of points having any possible shape.

Coords is implemented as a Numerical Python array with a length of its
last axis equal to 3.
Each set of 3 values along the last axis represents a single point in 3D.

The datatype should be a float type; default is Float.

\begin{classdesc}{Coords}{data=None,dtyp=None,copy=False}
Create a new instance of class Coords.

If no data are given, a single point (0.,0.,0.) will be created.
If specified, data should evaluate to an (...,3) shaped array of floats.
If copy==True, the data are copied.
If no dtype is given that of data are used, or float32 by default.

\end{classdesc}


\subsection{Methods returning information}

\begin{methoddesc}{points}{}
Return the data as a simple set of points.

This reshapes the array to a 2-dimensional array, flattening
the structure of the points.
\end{methoddesc}

    
\begin{methoddesc}{pshape}{}
Return shape of the points array.

This is the shape of the Coords array with last axis removed.

The full shape of the \class{Coords} array can be obtained from its \var{shape} attribute.\footnote{Note that this differs from the \class{Formex} class, where you have to use a call to the shape() method.} 
\end{methoddesc}


\begin{methoddesc}{npoints}{}
Return the total number of points.
\end{methoddesc}
\begin{methoddesc}{x}{}
  Return the x-plane.
\end{methoddesc}
\begin{methoddesc}{y}{}
  Return the y-plane.
\end{methoddesc}
\begin{methoddesc}{z}{}
  Return the z-plane.
\end{methoddesc}

\begin{methoddesc}{bbox}{}
Return the bounding box of a set of points.

        The bounding box is the smallest rectangular volume in global
        coordinates, such at no points are outside the box.
        It is returned as a Coords object with shape (2,3): the first row
        holds the minimal coordinates and the second row the maximal.
\end{methoddesc}

\begin{methoddesc}{center}{}
Return the center of the Coords.

        The center of a Coords is the center of its bbox().
        The return value is a (3,) shaped Coords object.\end{methoddesc}

\begin{methoddesc}{centroid}{}
Return the centroid of the Coords.

        The centroid of a Coords is the point whose coordinates
        are the mean values of all points.
        The return value is a (3,) shaped Coords object.\end{methoddesc}

\begin{methoddesc}{sizes}{}
Returns an array with shape (3,) holding the length of the bbox along the 3 axes.
\end{methoddesc}

\begin{methoddesc}{diagonal}{}
Return the length of the diagonal of the bbox().
\end{methoddesc}

\begin{methoddesc}{bsphere}{}
Return the diameter of the bounding sphere of the Coords.

The bounding sphere is the smallest sphere with center in the \function{center()} of the Coords, and such that no points of the Coords are lying outside the sphere. It is not necessarily the smallest sphere surrounding all points of the Coords.
\end{methoddesc}


\begin{methoddesc}{distanceFromPlane}{p,n}
    Return the distance from the plane (p,n) for all points of the Coords.

    p is a point specified by 3 coordinates.
    n is the normal vector to a plane, specified by 3 components.

    The return value is a [...] shaped array with the distance of
    each point to the plane containing the point  p and having normal n.
    Distance values are positive if the point is on the side of the
    plane indicated by the positive normal.
\end{methoddesc}


\begin{methoddesc}{distanceFromLine}{p,q}
    Return the distance from the line (p,q) for all points of the Coords.

    p and q are two points specified by 3 coordinates.

    The return value is a [...] shaped array with the distance of
    each point to the line through p and q.
    All distance values are positive or zero.
\end{methoddesc}


\begin{methoddesc}{distanceFromPoint}{p}
    Return the distance from the point p for all points of the Coords.

    p is a point specified by 3 coordinates.

    The return value is a [...] shaped array with the distance of
    each point to the line through p and q.
    All distance values are positive or zero.
\end{methoddesc}


\begin{methoddesc}{test}{dir=0,min=None,max=None}
  Flag points having coordinates between min and max.
  
This function is very convenient in clipping a Coords in a specified
direction. It returns a 1D integer array flagging (with a value 1 or
True) the elements having nodal coordinates in the required range.
Use \Code{where(result)} to get a list of element numbers passing the test.
You can also directly use \Code{x[x.test(...)]} to get the clipped Coords.
  
The test plane can be defined in two ways, depending on the value of dir.
If \var{dir} is a single integer (0, 1 or 2), it specifies a global axis
and \var{min} and \var{max} are the minimum and maximum values for the
coordinates along that axis.
Default is the 0 (or x) direction.

Else, \var{dir} should be compatible with a (3,) shaped array and specifies
the direction of the normal on the planes. In this case, \var{min} and \var{max}
are points and should also evaluate to (3,) shaped arrays.

One of the two clipping planes may be left unspecified.
\end{methoddesc}


\begin{methoddesc}{fprint}{fmt="\%10.3e \%10.3e \%10.3e"}
Prints all the points of the formex with the specified format. The format should hold three formatting codes, for the three coordinates of the point. 
\end{methoddesc}
 

\subsection{Coordinate transformations}

The following coordinate transforming methods are defined in the Coords class.
They implement the functionality of the \class{Formex} methods with the same name. We refer the reader to section~\ref{sec:formex} for a detailed description of these methods. 
\begin{methoddesc}{scale}{scale}
\end{methoddesc}

\begin{methoddesc}{translate}{dir,distance=None}
\end{methoddesc}

\begin{methoddesc}{rotate}{angle,axis=2}
\end{methoddesc}

\begin{methoddesc}{shear}{dir,dir1,skew}
\end{methoddesc}

\begin{methoddesc}{reflect}{dir,pos=0}
\end{methoddesc}

\begin{methoddesc}{mirror}{dir,pos=0}
\end{methoddesc}

\begin{methoddesc}{affine}{mat,vec=None}
\end{methoddesc}


\begin{methoddesc}{cylindrical}{dir=[0,1,2],scale=[1.,1.,1.]}
\end{methoddesc}

\begin{methoddesc}{toCylindrical}{dir=[0,1,2]}
\end{methoddesc}

\begin{methoddesc}{spherical}{dir=[0,1,2],scale=[1.,1.,1.],colat=False}
\end{methoddesc}

\begin{methoddesc}{toSpherical}{dir=[0,1,2]}
\end{methoddesc}

\begin{methoddesc}{bump}{dir,a,func,dist=None}
\end{methoddesc}

\begin{methoddesc}{bump1}{dir,a,func,dist}
\end{methoddesc}

\begin{methoddesc}{bump2}{dir,a,func}
\end{methoddesc}

\begin{methoddesc}{map}{func}
\end{methoddesc}

\begin{methoddesc}{map1}{dir,func}
\end{methoddesc}

\begin{methoddesc}{mapd}{dir,func,point,dist=None}
\end{methoddesc}

\begin{methoddesc}{replace}{i,j,other=None}
\end{methoddesc}

\begin{methoddesc}{swapaxes}{i,j}
\end{methoddesc}

\begin{methoddesc}{rollAxes}{n=1}
\end{methoddesc}

\begin{methoddesc}{projectOnSphere}{radius,center=[0.,0.,0.]}
\end{methoddesc}



\subsection{Other methods}


\begin{methoddesc}{fuse}{nodesperbox=1,shift=0.5,rtol=1.e-5,atol=1.e-5)}
Find (almost) identical nodes and return a compressed set.

This method finds the points that are very close and replaces them
with a single point. The return value is a tuple of two arrays:
- the unique points as a Coords object,
- an integer (nnod) array holding an index in the unique
coordinates array for each of the original nodes. This index will
have the same shape as the pshape() of the coords array.

The procedure works by first dividing the 3D space in a number of
equally sized boxes, with a mean population of nodesperbox.
The boxes are numbered in the 3 directions and a unique integer scalar
is computed, that is then used to sort the nodes.
Then only nodes inside the same box are compared on almost equal
coordinates, using the numpy allclose() function. Two coordinates are
considered close if they are within a relative tolerance rtol or absolute
tolerance atol. See numpy for detail. The default atol is set larger than
in numpy, because pyformex typically runs with single precision.
Close nodes are replaced by a single one.

Currently the procedure does not guarantee to find all close nodes:
two close nodes might be in adjacent boxes. The performance hit for
testing adjacent boxes is rather high, and the probability of separating
two close nodes with the computed box limits is very small. Nevertheless
we intend to access this problem by repeating the procedure with the
boxes shifted in space.
\end{methoddesc}


The following methods are class methods for the \class{Coords} class.
They should be called as \Code{Coords.method()}.

\begin{methoddesc}{concatenate}{L}
Concatenate a list of Coords object.

 All Coords object in the list L should have the same shape
 except for the length of the first axis.
 This function is equivalent to the numpy concatenate, but makes
 sure the result is a Cooords object.

\classmethod
\end{methoddesc}

\begin{methoddesc}{fromfile}{*args}
Read a Coords from file.

        This convenience function uses the numpy fromfile function to read
        the coordinates from file.
        You just have to make sure that the coordinates are read in order
        (X,Y,Z) for subsequent points, and that the total number of
        coordinates read is a multiple of 3.

\classmethod
\end{methoddesc}


\begin{methoddesc}{interpolate}{F,G,div}
Create interpolations between two Coords.

\var{F} and \var{G} are two Formices with the same shape.
\var{v} is a list of floating point values.
The result is the concatenation of the interpolations of \var{F} and \var{G} at all the values in \var{div}.

An interpolation of \var{F} and \var{G} at value \var{v} is a Formex \var{H} where each coordinate \var{Hijk} is obtained from  \Code{Hijk = Fijk + v * (Gijk-Fijk)}.
Thus, a Formex \Code{interpolate(F,G,[0.,0.5,1.0])} will contain all elements
of \var{F} and \var{G} and all elements with mean coordinates between those of \var{F} and \var{G}.

As a convenience, if an integer is specified for \var{div}, it is taken as a
number of division for the interval [0..1].
Thus, \Code{Coords.interpolate(F,G,n)} is equivalent with
\Code{Coords.interpolate(F,G,arange(0,n+1)/float(n))}

The resulting \class{Coords} array has an extra axis (the first). Its shape is
\Code{(n,) + F.shape}, where \Code{n} is the number of divisions.

\classmethod
\end{methoddesc}



\subsection{Other definitions in module \module{coords}}
The following variables and functions are defined in the file coords.py, 
but are not part of the \class{Coords} class. They are defined here either
because they are used in this module, or just to make them available to all
your scripts.

\begin{datadesc}{rad}
\Code{rad = pi/180}. This convenient definition provides easy conversion of angles from degrees to radians (\Code{*rad}) and back (\Code{/rad}).
\end{datadesc}

\begin{funcdesc}{sind}{arg}
    Return the sin of an angle in degrees.
\end{funcdesc}

\begin{funcdesc}{cosd}{arg}
    Return the cos of an angle in degrees.
\end{funcdesc}

\begin{funcdesc}{tand}{arg}
    Return the tan of an angle in degrees.
\end{funcdesc}

\begin{funcdesc}{length}{arg}
    Return the quadratic norm of a vector with all elements of arg.
\end{funcdesc}

\begin{funcdesc}{inside}{p,mi,ma}
    Return true if point p is inside bbox defined by points mi and ma
\end{funcdesc}

\begin{funcdesc}{isClose}{values,target,rtol=1.e-5,atol=1.e-8}
Return an array flagging the elements in values that are close to target.

\var{values} is a float array, \var{target} is a float value.
\var{values} and \var{target} should be broadcastable to the same shape.
    
The return value is a boolean array with shape of \var{values} flagging
the values that are close to target.
Two values \var{a} and \var{b}  are considered close if 
\Code{abs(a-b) < atol + rtol * abs(b)}
\end{funcdesc}


\begin{funcdesc}{unitVector}{axis}
Return a unit vector in the direction of a global axis (0,1,2).

Use normalize() to get a unit vector in a general direction.
\end{funcdesc}


\begin{funcdesc}{rotationMatrix}{angle,axis=None}
Return a rotation matrix over angle, optionally around axis.

The angle is specified in degrees.
If axis==None (default), a 2x2 rotation matrix is returned.
Else, axis should specifying the rotation axis in a 3D world. It is either
one of 0,1,2, specifying a global axis, or a vector with 3 components
specifying an axis through the origin.
In either case a 3x3 rotation matrix is returned.
Note that:
rotationMatrix(angle,[1,0,0]) == rotationMatrix(angle,0) 
rotationMatrix(angle,[0,1,0]) == rotationMatrix(angle,1) 
rotationMatrix(angle,[0,0,1]) == rotationMatrix(angle,2)
but the latter functions calls are more efficient.
The result is returned as an array.
\end{funcdesc}

   
\begin{funcdesc}{bbox}{objects}
Computes the overall bounding box of a list of objects.

The result has the same format as the \class{Coords} class \Code{bbox()} method, but the resulting box encloses all the objects in the list. All objects in the list should have the \var{bbox} method.
\end{funcdesc}


%%% Local Variables: 
%%% mode: latex
%%% TeX-master: "pyformex"
%%% End: 
