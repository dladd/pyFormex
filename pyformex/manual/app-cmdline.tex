% pyformex manual --- Appendix with the command line options
% $Id $


\chapter{Command line options}
\label{cha:commandline}

The following is a complete list of the options for the \Code{pyformex} command.This output can also be generated by the command \Code{pyformex {-}{-}help}.
\verbatiminput{pyformex.help}

\paragraph{Running \pyFormex without the GUI}
If you start \pyf with the \Code{{-}{-}nogui} option, no Graphical User Interface is created. This is extremely useful to run automated scripts in batch mode. In this operating mode, \pyf will interprete all arguments remaining after interpreting the options, as filenames of scripts to be run (and possibly arguments to be interpreted by these scripts).
Thus, if you want to run a \pyf script \Code{myscript.py} in batch mode, just give the command \Code{pyformex myscript.py}.

The running script has access to the remaining arguments in the global list variable \var{argv}. The script can use any arguments of it and pop them of the list. Any arguments remaining in the \var{argv} list when the script finishes, will be used for another \pyf execution cycle. This means that the first remaining argument should again be a \pyf script.
 
\endinput

%%% Local Variables: 
%%% mode: latex
%%% TeX-master: "pyformex"
%%% End: 
