% pyformex manual --- properties --- CREATED WITH py2ptex.py: DO NOT EDIT
% $Id$
% (C) B.Verhegghe

\section{\module{properties} --- General framework for attributing properties to geometrical elements.}
\label{sec:properties}

\declaremodule{""}{properties}
\modulesynopsis{General framework for attributing properties to geometrical elements.}
\moduleauthor{'pyFormex project'}{'http://pyformex.berlios.de'}

Properties can really be just about any Python object.
Properties can be attributed to a set of geometrical elements.



\subsection{Database class: A class for storing properties in a database.}




The Database class has this constructor: 

\begin{classdesc}{Database}{data={}}
Initialize a database.

        The database can be initialized with a dict.
        

\end{classdesc}

Database objects have the following methods:

\begin{funcdesc}{readDatabase}{filename,None}
Import all records from a database file.

        For now, it can only read databases using flatkeydb.
        args and kargs can be used to specify arguments for the
        FlatDB constructor.
        
\end{funcdesc}


\subsection{MaterialDB class: A class for storing material properties.}




The MaterialDB class has this constructor: 

\begin{classdesc}{MaterialDB}{data={}}
Initialize a materials database.

        If data is a dict, it contains the database.
        If data is a string, it specifies a filename where the
        database can be read.
        

\end{classdesc}

MaterialDB objects have the following methods:


\subsection{SectionDB class: A class for storing section properties.}




The SectionDB class has this constructor: 

\begin{classdesc}{SectionDB}{data={}}
Initialize a section database.

        If data is a dict, it contains the database.
        If data is a string, it specifies a filename where the
        database can be read.
        

\end{classdesc}

SectionDB objects have the following methods:


\subsection{ElemSection class: Properties related to the section of an element.}




The ElemSection class has this constructor: 

\begin{classdesc}{ElemSection}{section=None,material=None,orientation=None,behavior=None}
Create a new element section property. Empty by default.
        
        An element section property can hold the following sub-properties:
       - section: the section properties of the element. This can be a dict
          or a string. The required data in this dict depend on the
          sectiontype. Currently the following keys are used by fe_abq.py:
            - sectiontype: the type of section: one of following:
              'solid': a solid 2D or 3D section,
              'circ' : a plain circular section,
              'rect' : a plain rectangular section,
              'pipe' : a hollow circular section,
              'box'  : a hollow rectangular section,
              'I'    : an I-beam,
              'general' : anything else (automatically set if not specified).
              !! Currently only 'solid' and 'general' are allowed.
            - the cross section characteristics :
              cross_section, moment_inertia_11, moment_inertia_12,
              moment_inertia_22, torsional_rigidity
            - for sectiontype 'circ': radius
         - material: the element material. This can be a dict or a string.
          Currently known keys to fe_abq.py are:
            young_modulus, shear_modulus, density, poisson_ratio
        - 'orientation' is a list of 3 direction cosines of the first beam
          section axis.
        - behavior: the behavior of the connector
        

\end{classdesc}

ElemSection objects have the following methods:

\begin{funcdesc}{addSection}{section}
Create or replace the section properties of the element.

        If 'section' is a dict, it will be added to 'self.secDB'.
        If 'section' is a string, this string will be used as a key to
        search in 'self.secDB'.
        
\end{funcdesc}

\begin{funcdesc}{computeSection}{section}
Compute the section characteristics of specific sections.
\end{funcdesc}

\begin{funcdesc}{addMaterial}{material}
Create or replace the material properties of the element.

        If the argument is a dict, it will be added to 'self.matDB'.
        If the argument is a string, this string will be used as a key to
        search in 'self.matDB'.
        
\end{funcdesc}


\subsection{ElemLoad class: Distributed loading on an element.}




The ElemLoad class has this constructor: 

\begin{classdesc}{ElemLoad}{label=None,value=None}
Create a new element load. Empty by default.
        
        An element load can hold the following sub-properties:
        - label: the distributed load type label.
        - value: the magnitude of the distibuted load.
        

\end{classdesc}

ElemLoad objects have the following methods:


\subsection{CoordSystem class: A class for storing coordinate systems.}




The CoordSystem class has this constructor: 

\begin{classdesc}{CoordSystem}{csys,cdata}
Create a new coordinate system.

        csys is one of 'Rectangular', 'Spherical', 'Cylindrical'. Case is
          ignored and the first letter suffices.
        cdata is a list of 6 coordinates specifying the two points that
          determine the coordinate transformation 
        

\end{classdesc}

CoordSystem objects have the following methods:


\subsection{Amplitude class: A class for storing an amplitude.}

    The amplitude is a list of tuples (time,value).
    


The Amplitude class has this constructor: 

\begin{classdesc}{Amplitude}{data,definition='TABULAR'}
Create a new amplitude.

\end{classdesc}

Amplitude objects have the following methods:


\subsection{PropertyDB class: A database class for all properties.}

    This class collects all properties that can be set on a
    geometrical model.

    This should allow for storing:
       - materials
       - sections
       - any properties
       - node properties
       - elem properties
       - model properties (current unused: use unnamed properties)
    


The PropertyDB class has this constructor: 

\begin{classdesc}{PropertyDB}{}
Create a new properties database.

\end{classdesc}

PropertyDB objects have the following methods:

\begin{funcdesc}{autoName}{clas,kind}

\classmethod
\end{funcdesc}

\begin{funcdesc}{setMaterialDB}{aDict}
Set the materials database to an external source
\end{funcdesc}

\begin{funcdesc}{setSectionDB}{aDict}
Set the sections database to an external source
\end{funcdesc}

\begin{funcdesc}{Prop}{kind='',tag=None,set=None,setname=None}
Create a new property, empty by default.

        A property can hold almost anything, just like any Dict type.
        It has however four predefined keys that should not be used for
        anything else than explained hereafter:
        - nr: a unique id, that never should be set/changed by the user.
        - tag: an identification tag used to group properties
        - set: a single number or a list of numbers identifying the geometrical
               elements for wich the property is set, or the name of a
               previously defined set.
        - setname: the name to be used for this set. Default is to use an
               automatically generated name. If setname is specified without
               a set, this is interpreted as a set= field.
        Besides these, any other fields may be defined and will be added
        without checking.
        
\end{funcdesc}

\begin{funcdesc}{getProp}{kind='',rec=None,tag=None,attr=[],delete=False}
Return all properties of type kind matching tag and having attr.

        kind is either '', 'n', 'e' or 'm'
        If rec is given, it is a list of record numbers or a single number.
        If a tag or a list of tags is given, only the properties having a
        matching tag attribute are returned.
        If a list of attibutes is given, only the properties having those
        attributes are returned.

        If delete==True, the returned properties are removed from the database.
        
\end{funcdesc}

\begin{funcdesc}{delete}{plist,kind=''}
Delete properties in list pdel from list plist.
\end{funcdesc}

\begin{funcdesc}{sanitize}{kind}
Sanitize the record numbers after deletion
\end{funcdesc}

\begin{funcdesc}{delProp}{kind='',rec=None,tag=None,attr=[]}
Delete properties.

        This is equivalent to getProp() but the returned properties
        are removed from the database.
        
\end{funcdesc}

\begin{funcdesc}{nodeProp}{prop=None,set=None,setname=None,tag=None,cload=None,bound=None,displ=None,csys=None,ampl=None}
Create a new node property, empty by default.

        A node property can contain any combination of the following fields:
        - tag: an identification tag used to group properties (this is e.g.
               used to flag Step, increment, load case, ...)
        - set: a single number or a list of numbers identifying the node(s)
                for which this property will be set, or a set name
                If None, the property will hold for all nodes.
        - cload: a concentrated load: a list of 6 values
        - bound: a boundary condition: a list of 6 codes (0/1)
        - displ: a prescribed displacement: a list of tuples (dofid,value)
        - csys: a CoordSystem
        - ampl: the name of an Amplitude
        
\end{funcdesc}

\begin{funcdesc}{elemProp}{prop=None,grp=None,set=None,setname=None,tag=None,section=None,eltype=None,dload=None,ampl=None}
Create a new element property, empty by default.
        
        An elem property can contain any combination of the following fields:
        - tag: an identification tag used to group properties (this is e.g.
               used to flag Step, increment, load case, ...)
        - set: a single number or a list of numbers identifying the element(s)
                for which this property will be set, or a set name
                If None, the property will hold for all elements.
        - grp: an elements group number (default None). If specified, the
               element numbers given in set are local to the specified group.
               If not, elements are global and should match the global numbering
               according to the order in which element groups will be specified
               in the Model.
        - eltype: the element type (currently in Abaqus terms). 
        - section: an ElemSection specifying the element section properties.
        - dload: an ElemLoad specifying a distributed load on the element.
        - ampl: the name of an Amplitude
        
\end{funcdesc}


\subsection{Functions defined in the properties module}



\begin{funcdesc}{checkIdValue}{values}
Check that a variable is a list of values or (id,value) tuples

    If ok, return the values as a list of tuples.
    

\end{funcdesc}


\begin{funcdesc}{checkString}{a,valid}
Check that a string a has one of the valid values.

    This is case insensitive, and returns the upper case string if valid.
    Else, an error is raised.
    

\end{funcdesc}


\begin{funcdesc}{autoName}{base}


\end{funcdesc}


\begin{funcdesc}{Nset}{}


\end{funcdesc}


\begin{funcdesc}{Eset}{}


\end{funcdesc}


%%% Local Variables: 
%%% mode: latex
%%% TeX-master: "pyformex"
%%% End:

