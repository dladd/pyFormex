% pyformex manual --- isopar --- CREATED WITH py2ptex.py: DO NOT EDIT
% $Id$
% (C) B.Verhegghe

\section{\module{isopar} --- Isoparametric transformations}
\label{sec:isopar}

\declaremodule{""}{isopar}
\modulesynopsis{Isoparametric transformations}
\moduleauthor{'pyFormex project'}{'http://pyformex.berlios.de'}




\subsection{Isopar class: A class representing an isoparametric transformation}




The Isopar class has this constructor: 

\begin{classdesc}{Isopar}{eltype,coords,oldcoords}
Create an isoparametric transformation.

        type is one of the keys in Isopar.isodata
        coords and oldcoords can be either arrays, Coords or Formex instances,
        but should be of equal shape, and match the number of atoms in the
        specified transformation type
        

\end{classdesc}

Isopar objects have the following methods:

\begin{funcdesc}{transform}{X}
Apply isoparametric transform to a set of coordinates.

        Returns a Coords array with same shape as X
        
\end{funcdesc}

\begin{funcdesc}{transformFormex}{F}
Apply an isoparametric transform to a Formex.

        The result is a topologically equivalent Formex.
        
\end{funcdesc}


\subsection{Functions defined in the isopar module}



\begin{funcdesc}{build_matrix}{atoms,x,y=0,z=0}
Build a matrix of functions of coords.

    Atoms is a list of text strings representing some function of
    x(,y)(,z). x is a list of x-coordinats of the nodes, y and z can be set
    to lists of y,z coordinates of the nodes.
    Each line of the returned matrix contains the atoms evaluated at a
    node.
    

\end{funcdesc}


\begin{funcdesc}{transformFormex}{F,trf}


\end{funcdesc}


\begin{funcdesc}{isopar}{F,eltype,coords,oldcoords}


\end{funcdesc}


%%% Local Variables: 
%%% mode: latex
%%% TeX-master: "pyformex"
%%% End:

