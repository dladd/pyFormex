% POPPING (260, (259, (50, '@'), (287, (1, 'deprecation')), (7, '('), (329, (330, (303, (304, (305, (306, (307, (309, (310, (311, (312, (313, (314, (315, (316, (317, (3, '"\\nUse mesh.connectMesh instead."'))))))))))))))))), (8, ')'), (4, '')))
% MATCHING (259, (50, '@'), (287, (1, 'deprecation')))
% POPPING (260, (259, (50, '@'), (287, (1, 'deprecation')), (7, '('), (329, (330, (303, (304, (305, (306, (307, (309, (310, (311, (312, (313, (314, (315, (316, (317, (3, '"\\nUse mesh.connectMesh instead."'))))))))))))))))), (8, ')'), (4, '')))
% POPPING (260, (259, (50, '@'), (287, (1, 'deprecation')), (7, '('), (329, (330, (303, (304, (305, (306, (307, (309, (310, (311, (312, (313, (314, (315, (316, (317, (3, '"\\nUse mesh.sweepMesh instead."'))))))))))))))))), (8, ')'), (4, '')))
% MATCHING (259, (50, '@'), (287, (1, 'deprecation')))
% POPPING (260, (259, (50, '@'), (287, (1, 'deprecation')), (7, '('), (329, (330, (303, (304, (305, (306, (307, (309, (310, (311, (312, (313, (314, (315, (316, (317, (3, '"\\nUse mesh.sweepMesh instead."'))))))))))))))))), (8, ')'), (4, '')))
% pyformex manual --- mesh --- CREATED WITH py2ptex.py: DO NOT EDIT
% $Id$
% (C) B.Verhegghe

\section{\module{mesh} --- mesh.py}
\label{sec:mesh}

\declaremodule{""}{mesh}
\modulesynopsis{mesh.py}
\moduleauthor{'pyFormex project'}{'http://pyformex.berlios.de'}

Definition of the Mesh class for describing discrete geometrical models.
And some useful meshing functions to create such models.



\subsection{Mesh class: A mesh is a discrete geometrical model consisting of nodes and elements.}

    In the Mesh geometrical data model, coordinates of all points are gathered
    in a single twodimensional array 'coords' with shape (ncoords,3) and the
    individual geometrical elements are described by indices into the 'coords'
    array.
    This model has some advantages over the Formex data model, where all
    points of all element are stored by their coordinates:
    - compacter storage, because coordinates of coinciding points do not
    need to be repeated,
    - faster connectivity related algorithms.
    The downside is that geometry generating algorithms are far more complex
    and possibly slower.
    
    In pyFormex we therefore mostly use the Formex data model when creating
    geometry, but when we come to the point of exporting the geometry to
    file (and to other programs), a Mesh data model may be more adequate.

    The Mesh data model has at least the following attributes:
    coords: (ncoords,3) shaped Coords array,
    elems:  (nelems,nplex) shaped array of int32 indices into coords. All
                values should be in the range 0 <= value < ncoords.
    prop: array of element property numbers, default None.
    eltype: string designing the element type, default None.
    


The Mesh class has this constructor: 

\begin{classdesc}{Mesh}{coords=None,elems=None,prop=None,eltype=None}
Create a new Mesh from the specified data.

        data is either a tuple of (coords,elems) arrays, or an object having
        a 'toMesh()' method, which should return such a tuple.
        

\end{classdesc}

Mesh objects have the following methods:

\begin{funcdesc}{copy}{}
Return a copy using the same data arrays
\end{funcdesc}

\begin{funcdesc}{toFormex}{}
Convert a Mesh to a Formex.

        The Formex inherits the element property numbers and eltype from
        the Mesh. Node property numbers however can not be translated to
        the Formex data model.
        
\end{funcdesc}

\begin{funcdesc}{data}{}
Return the mesh data as a tuple (coords,elems)
\end{funcdesc}

\begin{funcdesc}{nelems}{}

\end{funcdesc}

\begin{funcdesc}{nplex}{}

\end{funcdesc}

\begin{funcdesc}{ncoords}{}

\end{funcdesc}

\begin{funcdesc}{shape}{}

\end{funcdesc}

\begin{funcdesc}{report}{}

\end{funcdesc}

\begin{funcdesc}{compact}{}
Renumber the mesh and remove unconnected nodes.
\end{funcdesc}

\begin{funcdesc}{extrude}{n,step=1.,dir=0,autofix=True}
Extrude a Mesh in one of the axes directions.

        Returns a new Mesh obtained by extruding the given Mesh
        over n steps of length step in direction of axis dir.
        The returned Mesh has double plexitude of the original.

        This function is usually used to extrude points into lines,
        lines into surfaces and surfaces into volumes.
        By default it will try to fix the connectivity ordering where
        appropriate. If autofix is switched off, the connectivities
        are merely stacked, and the user may have to fix it himself.

        Currently, this function correctly transforms: point1 to line2,
        line2 to quad4, tri3 to wedge6, quad4 to hex8.
        
\end{funcdesc}

\begin{funcdesc}{sweep}{path,eltype=None}
Sweep a mesh along a path, creating an extrusion
\end{funcdesc}

\begin{funcdesc}{convert}{fromtype,totype}
Convert a mesh from element type fromtype to type totype.

        Currently defined conversions:
        'quad4' -> 'tri3'
        
\end{funcdesc}

\begin{funcdesc}{concatenate}{clas,ML}
Concatenate a list of meshes of the same plexitude and eltype
\classmethod
\end{funcdesc}


\subsection{Functions defined in the mesh module}



\begin{funcdesc}{sweepCoords}{path,origin=[0.,0.,0.],normal=0,avgdir=False,enddir=None}
 Sweep a Coords object along a path, returning a series of copies.

    origin and normal define the local path position and direction on the mesh.
    
    At each point of the curve, a copy of the Coords object is created, with
    its origin in the curve's point, and its normal along the curve's direction.
    In case of a PolyLine, directions are pointing to the next point by default.
    If avgdir==True, average directions are taken at the intermediate points.
    Missing end directions can explicitely be set by enddir, and are by default
    taken along the last segment.
    If the curve is closed, endpoints are treated as any intermediate point,
    and the user should normally not specify enddir. 

    The return value is a sequence of the transformed Coords objects.
    

\end{funcdesc}


\begin{funcdesc}{connectMesh}{mesh1,mesh2,n=1,n1=None,n2=None,eltype=None}
Connect two meshes to form a hypermesh.
    
    mesh1 and mesh2 are two meshes with same topology (shape). 
    The two meshes are connected by a higher order mesh with n
    elements in the direction between the two meshes.
    n1 and n2 are node selection indices permitting a permutation of the
    nodes of the base sets in their appearance in the hypermesh.
    This can e.g. be used to achieve circular numbering of the hypermesh.
    

\end{funcdesc}


\begin{funcdesc}{connectMeshSequence}{ML,loop=False}


\end{funcdesc}


\begin{funcdesc}{createWedgeElements}{S1,S2,div=1}
Create wedge elements between to triangulated surfaces.
    
    6-node wedge elements are created between two input surfaces (S1 and S2).
    The keyword div determines the number of created wedge element layers.
    Layers with equal thickness are created when an integer value is used for div.
    div can also be specified using a list, that defines the interpolation between the two surfaces.
    Consequently, this can be used to create layers with unequal thickness.
    For example, div=2 gives the same result as [0.,0.5,1.]
    

\end{funcdesc}


\begin{funcdesc}{sweepGrid}{nodes,elems,path,scale=1.,angle=0.,a1=None,a2=None}
 Sweep a quadrilateral mesh along a path
    
    The path should be specified as a (n,2,3) Formex.
    The input grid (quadrilaterals) has to be specified with the nodes and elems and 
    can for example be created with the functions gridRectangle or gridBetween2Curves.
    This quadrilateral grid should be within the YZ-plane.
    The quadrilateral grid can be scaled and/or rotated along the path.
    
    There are three options for the first (a1) / last (a2) element of the path:
    1) None: No corresponding hexahedral elements
    2) 'last': The direction of the first/last element of the path is used to 
    direct the input grid at the start/end of the path
    3) specify a vector: This vector is used to direct the input grid at the start/end of the path
    
    The resulting hexahedral mesh is returned in terms of nodes and elems.
    

\end{funcdesc}


%%% Local Variables: 
%%% mode: latex
%%% TeX-master: "pyformex"
%%% End:

