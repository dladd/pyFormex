% pyformex manual --- mesh --- CREATED WITH py2ptex.py: DO NOT EDIT
% $Id$
% (C) B.Verhegghe

\section{\module{mesh} --- mesh.py}
\label{sec:mesh}

\declaremodule{""}{mesh}
\modulesynopsis{mesh.py}
\moduleauthor{'pyFormex project'}{'http://pyformex.berlios.de'}

A plugin providing some useful meshing functions.



\begin{funcdesc}{createWedgeElements}{S1,S2,div=1}
Create wedge elements between to triangulated surfaces.
    
    6-node wedge elements are created between two input surfaces (S1 and S2).
    The keyword div determines the number of created wedge element layers.
    Layers with equal thickness are created when an integer value is used for div.
    div can also be specified using a list, that defines the interpolation between the two surfaces.
    Consequently, this can be used to create layers with unequal thickness.
    For example, div=2 gives the same result as [0.,0.5,1.]
    

\end{funcdesc}


\begin{funcdesc}{sweepGrid}{nodes,elems,path,scale=1.,angle=0.,a1=None,a2=None}
 Sweep a quadrilateral mesh along a path
    
    The path should be specified as a (n,2,3) Formex.
    The input grid (quadrilaterals) has to be specified with the nodes and elems and 
    can for example be created with the functions gridRectangle or gridBetween2Curves.
    This quadrilateral grid should be within the YZ-plane.
    The quadrilateral grid can be scaled and/or rotated along the path.
    
    There are three options for the first (a1) / last (a2) element of the path:
    1) None: No corresponding hexahedral elements
    2) 'last': The direction of the first/last element of the path is used to 
    direct the input grid at the start/end of the path
    3) specify a vector: This vector is used to direct the input grid at the start/end of the path
    
    The resulting hexahedral mesh is returned in terms of nodes and elems.
    

\end{funcdesc}


\begin{funcdesc}{connectMesh}{coords1,coords2,elems,n=1,n1=None,n2=None}
Connect two meshes to form a hypermesh.
    
    coords1,elems and coords2,elems are 2 meshes with same topology.
    The coordinates are given in corresponding order.
    The two meshes are connected by a higher order mesh with n
    elements in the direction between the two meshes.
    n1 and n2 are node selection indices permitting a permutation of the
    nodes of the base sets in their appearance in the hypermesh.
    This can e.g. be used to achieve circular numbering of the hypermesh.
    

\end{funcdesc}


\begin{funcdesc}{extrudeMesh}{coords,elems,n,step=1.,dir=0,autofix=True}
Extrude a mesh in one of the axes directions.

    Returns a hypermesh obtained by extruding the given mesh (coords,elems)
    over n steps of length step in direction of axis dir.
    The returned mesh has double plexitude of the original.

    This function is usually used to points into lines, lines into surfaces
    and surfaces into volumes. By default it will try to fix the connectivity
    ordering where appropriate. It autofix is switched off, the connectivities
    are merely stacked, and the user may have to fix it himself.

    Currently, this function correctly transforms: point1 to line2,
    line2 to quad4, tri3 to wedge6, quad4 to hex8.
    

\end{funcdesc}


%%% Local Variables: 
%%% mode: latex
%%% TeX-master: "pyformex"
%%% End:

