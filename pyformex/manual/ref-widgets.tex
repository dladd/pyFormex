% pyformex manual --- widgets
% $Id$
% (C) B.Verhegghe


\section{\module{widgets} --- A collection of custom widgets used in the pyFormex GUI}
\label{sec:widgets}

\declaremodule{""}{pyformex/gui/widgets}
\modulesynopsis{pyformex/gui/widgets}
\moduleauthor{'pyFormex project'}{'http://pyformex.org'}



\subsection{Options --- }


\begin{classdesc}{Options}{}

\end{classdesc}

\subsection{FileSelection --- A file selection dialog widget.}
    You can specify a default path/filename that will be suggested initially.
    If a pattern is specified, only matching files will be shown.
    A pattern can be something like 'Images (*.png *.jpg)' or a list
    of such strings.
    Default mode is to accept any filename. You can specify exist=True
    to accept only existing files. Or set exist=True and multi=True to
    accept multiple files.
    If dir==True, a single existing directory is asked.
    

\begin{classdesc}{FileSelection}{path,pattern=None,exist=False,multi=False,dir=False}
The constructor shows the widget.
\end{classdesc}

FileSelection instances have the following methods:

\begin{funcdesc}{getFilename}{}
Ask for a filename by user interaction.

        Return the filename selected by the user.
        If the user hits CANCEL or ESC, None is returned.
        
\end{funcdesc}

\subsection{SaveImageDialog --- A file selection dialog with extra fields.}


\begin{classdesc}{SaveImageDialog}{path=None,pattern=None,exist=False,multi=False}
Create the dialog.
\end{classdesc}

SaveImageDialog instances have the following methods:

\begin{funcdesc}{getResult}{}

\end{funcdesc}

\subsection{ImageViewerDialog --- }


\begin{classdesc}{ImageViewerDialog}{path=None}

\end{classdesc}

ImageViewerDialog instances have the following methods:

\begin{funcdesc}{getFilename}{}
Ask for a filename by user interaction.

        Return the filename selected by the user.
        If the user hits CANCEL or ESC, None is returned.
        
\end{funcdesc}

\subsection{AppearenceDialog --- A dialog for setting the GUI appearance.}


\begin{classdesc}{AppearenceDialog}{}
Create the Appearance dialog.
\end{classdesc}

AppearenceDialog instances have the following methods:

\begin{funcdesc}{setFont}{}

\end{funcdesc}

\begin{funcdesc}{getResult}{}

\end{funcdesc}

\subsection{DockedSelection --- A widget that is docked in the main window and contains a modeless}
    

\begin{classdesc}{DockedSelection}{slist=[],title='Selection Dialog',mode=None,sort=False,func=None}

\end{classdesc}

DockedSelection instances have the following methods:

\begin{funcdesc}{setSelected}{selected,bool}

\end{funcdesc}

\begin{funcdesc}{getResult}{}

\end{funcdesc}

\subsection{ModelessSelection --- A modeless dialog for selecting one or more items from a list.}


\begin{classdesc}{ModelessSelection}{slist=[],title='Selection Dialog',mode=None,sort=False,func=None}
Create the SelectionList dialog.
        
\end{classdesc}

ModelessSelection instances have the following methods:

\begin{funcdesc}{setSelected}{selected,bool}
Mark the specified items as selected.
\end{funcdesc}

\begin{funcdesc}{getResult}{}
Return the list of selected values.

        If the user cancels the selection operation, the return value is None.
        Else, the result is always a list, possibly empty or with a single
        value.
        
\end{funcdesc}

\subsection{Selection --- A dialog for selecting one or more items from a list.}


\begin{classdesc}{Selection}{slist=[],title='Selection Dialog',mode=None,sort=False,selected=[]}
Create the SelectionList dialog.

        selected is a list of items that are initially selected.
        
\end{classdesc}

Selection instances have the following methods:

\begin{funcdesc}{setSelected}{selected}
Mark the specified items as selected.
\end{funcdesc}

\begin{funcdesc}{getResult}{}
Return the list of selected values.

        If the user cancels the selection operation, the return value is None.
        Else, the result is always a list, possibly empty or with a single
        value.
        
\end{funcdesc}

\subsection{InputItem --- A single input item, usually with a label in front.}
    The created widget is a QHBoxLayout which can be embedded in the vertical
    layout of a dialog.
    
    This is a super class, which just creates the label. The input
    field(s) should be added by a dedicated subclass.

    This class also defines default values for the name() and value()
    methods.

    Subclasses should override:
    - name(): if they called the superclass __init__() method without a name;
    - value(): if they did not create a self.input widget who's text() is
      the return value of the item.
    - setValue(): always, unless the field is readonly.

    Subclases can set validators on the input, like
      input.setValidator(QtGui.QIntValidator(input))
    Subclasses can define a show() method e.g. to select the data in the
    input field on display of the dialog.
    

\begin{classdesc}{InputItem}{name=None,*args}
Creates a new inputitem with a name label in front.
        
        If a name is given, a label is created and added to the layout.
        
\end{classdesc}

InputItem instances have the following methods:

\begin{funcdesc}{name}{}
Return the widget's name.
\end{funcdesc}

\begin{funcdesc}{value}{}
Return the widget's value.
\end{funcdesc}

\begin{funcdesc}{setValue}{val}
Change the widget's value.
\end{funcdesc}

\subsection{InputInfo --- An unchangeable input item.}


\begin{classdesc}{InputInfo}{name,value,*args}
Creates a new info field with a label in front.

        The info input field is an unchangeable text field.
        
\end{classdesc}

InputInfo instances have the following methods:

\begin{funcdesc}{value}{}
Return the widget's value.
\end{funcdesc}

\begin{funcdesc}{setValue}{val}
Change the widget's value.
\end{funcdesc}

\subsection{InputString --- A string input item.}


\begin{classdesc}{InputString}{name,value,*args}
Creates a new string input field with a label in front.

        If the type of value is not a string, the input string
        will be eval'ed before returning.
        
\end{classdesc}

InputString instances have the following methods:

\begin{funcdesc}{show}{}
Select all text on first display.
\end{funcdesc}

\begin{funcdesc}{value}{}
Return the widget's value.
\end{funcdesc}

\begin{funcdesc}{setValue}{val}
Change the widget's value.
\end{funcdesc}

\subsection{InputBool --- A boolean input item.}


\begin{classdesc}{InputBool}{name,value,*args}
Creates a new checkbox for the input of a boolean value.
        
        Displays the name next to a checkbox, which will initially be set
        if value evaluates to True. (Does not use the label)
        The value is either True or False,depending on the setting
        of the checkbox.
        
\end{classdesc}

InputBool instances have the following methods:

\begin{funcdesc}{name}{}
Return the widget's name.
\end{funcdesc}

\begin{funcdesc}{value}{}
Return the widget's value.
\end{funcdesc}

\begin{funcdesc}{setValue}{val}
Change the widget's value.
\end{funcdesc}

\subsection{InputCombo --- A combobox InputItem.}


\begin{classdesc}{InputCombo}{name,choices,default,*args}
Creates a new combobox for the selection of a value from a list.

        choices is a list/tuple of possible values.
        default is the initial/default choice.
        If default is not in the choices list, it is prepended.
        If default is None, the first item of choices is taken as the default.
       
        The choices are presented to the user as a combobox, which will
        initially be set to the default value.
        
\end{classdesc}

InputCombo instances have the following methods:

\begin{funcdesc}{value}{}
Return the widget's value.
\end{funcdesc}

\begin{funcdesc}{setValue}{val}
Change the widget's value.
\end{funcdesc}

\subsection{InputRadio --- A radiobuttons InputItem.}


\begin{classdesc}{InputRadio}{name,choices,default,direction='h',*args}
Creates radiobuttons for the selection of a value from a list.

        choices is a list/tuple of possible values.
        default is the initial/default choice.
        If default is not in the choices list, it is prepended.
        If default is None, the first item of choices is taken as the default.
       
        The choices are presented to the user as a hbox with radio buttons,
        of which the default will initially be pressed.
        If direction == 'v', the options are in a vbox. 
        
\end{classdesc}

InputRadio instances have the following methods:

\begin{funcdesc}{value}{}
Return the widget's value.
\end{funcdesc}

\begin{funcdesc}{setValue}{val}
Change the widget's value.
\end{funcdesc}

\subsection{InputPush --- A pushbuttons InputItem.}


\begin{classdesc}{InputPush}{name,choices,default=None,direction='h',*args}
Creates pushbuttons for the selection of a value from a list.

        choices is a list/tuple of possible values.
        default is the initial/default choice.
        If default is not in the choices list, it is prepended.
        If default is None, the first item of choices is taken as the default.
       
        The choices are presented to the user as a hbox with radio buttons,
        of which the default will initially be pressed.
        If direction == 'v', the options are in a vbox. 
        
\end{classdesc}

InputPush instances have the following methods:

\begin{funcdesc}{setText}{text,index=0}
Change the text on button index.
\end{funcdesc}

\begin{funcdesc}{setIcon}{icon,index=0}
Change the icon on button index.
\end{funcdesc}

\begin{funcdesc}{value}{}
Return the widget's value.
\end{funcdesc}

\begin{funcdesc}{setValue}{val}
Change the widget's value.
\end{funcdesc}

\subsection{InputInteger --- An integer input item.}
    Options:
    'min', 'max': range of the scale (integer)
    

\begin{classdesc}{InputInteger}{name,value,*args,**kargs}
Creates a new integer input field with a label in front.
\end{classdesc}

InputInteger instances have the following methods:

\begin{funcdesc}{show}{}
Select all text on first display.
\end{funcdesc}

\begin{funcdesc}{value}{}
Return the widget's value.
\end{funcdesc}

\begin{funcdesc}{setValue}{val}
Change the widget's value.
\end{funcdesc}

\subsection{InputFloat --- An float input item.}


\begin{classdesc}{InputFloat}{name,value,*args}
Creates a new float input field with a label in front.
\end{classdesc}

InputFloat instances have the following methods:

\begin{funcdesc}{show}{}
Select all text on first display.
\end{funcdesc}

\begin{funcdesc}{value}{}
Return the widget's value.
\end{funcdesc}

\begin{funcdesc}{setValue}{val}
Change the widget's value.
\end{funcdesc}

\subsection{InputSlider --- An integer input item using a slider.}
    Options:
      'min', 'max': range of the scale (integer)
      'ticks'     : step for the tick marks (default range length / 10)
      'func'      : an optional function to be called whenever the value is
                    changed. The function takes a float/integer argument.
    

\begin{classdesc}{InputSlider}{name,value,*args,**kargs}
Creates a new integer input slider.
\end{classdesc}

InputSlider instances have the following methods:

\begin{funcdesc}{set_value}{val}

\end{funcdesc}

\subsection{InputFSlider --- A float input item using a slider.}
    Options:
      'min', 'max': range of the scale (integer)
      'ticks'     : step for the tick marks (default range length / 10)
      'func'      : an optional function to be called whenever the value is
                    changed. The function takes a float/integer argument.
    

\begin{classdesc}{InputFSlider}{name,value,*args,**kargs}
Creates a new integer input slider.
\end{classdesc}

InputFSlider instances have the following methods:

\begin{funcdesc}{set_value}{val}

\end{funcdesc}

\subsection{InputColor --- A color input item.}


\begin{classdesc}{InputColor}{name,value,*args}
Creates a new color input field with a label in front.

        The color input field is a button displaying the current color.
        Clicking on the button opens a color dialog, and the returned
        value is set in the button.
        
\end{classdesc}

InputColor instances have the following methods:

\begin{funcdesc}{setColor}{}

\end{funcdesc}

\begin{funcdesc}{setValue}{value}
Change the widget's value.
\end{funcdesc}

\subsection{InputDialog --- A dialog widget to set the value of one or more items.}
    While general input dialogs can be constructed from all the underlying
    Qt classes, this widget provides a way to construct fairly complex
    input dialogs with a minimum of effort.
    

\begin{classdesc}{InputDialog}{items,caption=None,parent=None,flags=None,actions=None,default=None,report_pos=False}
Creates a dialog which asks the user for the value of items.

        Each item in the 'items' list is a tuple holding at least the name
        of the item, and optionally some more elements that limit the type
        of data that can be entered. The general format of an item is:
          name,value,type,range
        It should fit one of the following schemes:
        ('name',str) : type string, any string input allowed
        ('name',int) : type int, any integer value allowed
        ('name',int,'min','max') : type int, only min <= value <= max allowed
        For each item a label with the name and a LineEdit widget are created,
        with a validator function where appropriate.

        Input items are defined by a list with the following structure:
        [ name, value, type, range... ]
        The fields have the following meaning:
          name:  the name of the field,
          value: the initial or default value of the field,
          type:  the type of values the field can accept,
          range: the range of values the field can accept,
        The first two fields are mandatory. In many cases the type can be
        determined from the value and no other fields are required. Thus:
        [ 'name', 'value' ] will accept any string (initial string = 'value'),
        [ 'name', True ] will show a checkbox with the item checked,
        [ 'name', 10 ] will accept any integer,
        [ 'name', 1.5 ] will accept any float.

        Range settings for int and float types:
        [ 'name', 1, int, 0, 4 ] will accept an integer from 0 to 4, inclusive;
        [ 'name', 1, float, 0.0, 1.0, 2 ] will accept a float in the range
           from 0.0 to 1.0 with a maximum of two decimals.

        Composed types:
        [ 'name', 'option1', 'select', ['option0','option1','option2']] will
        present a combobox to select between one of the options.
        The initial and default value is 'option1'.

        [ 'name', 'option1', 'radio', ['option0','option1','option2']] will
        present a group of radiobuttons to select between one of the options.
        The initial and default value is 'option1'.
        A variant 'vradio' aligns the options vertically. 
        
        [ 'name', 'option1', 'push', ['option0','option1','option2']] will
        present a group of pushbuttons to select between one of the options.
        The initial and default value is 'option1'.
        A variant 'vpush' aligns the options vertically. 

        [ 'name', 'red', 'color' ] will present a color selection widget,
        with 'red' as the initial choice.
        
\end{classdesc}

InputDialog instances have the following methods:

\begin{funcdesc}{__getitem__}{name}
Return the input item with specified name.
\end{funcdesc}

\begin{funcdesc}{acceptData}{}
Update the dialog's return value from the field values.

        This function is connected to the 'accepted()' signal.
        Modal dialogs should normally not need to call it.
        In non-modal dialogs however, you can call it to update the
        results without having to raise the accepted() signal (which
        would close the dialog).
        
\end{funcdesc}

\begin{funcdesc}{updateData}{d}
Update a dialog from the data in given dictionary.

        d is a dictionary where the keys are field names in t the dialog.
        The values will be set in the corresponding input items.
        
\end{funcdesc}

\begin{funcdesc}{timeout}{}

\end{funcdesc}

\begin{funcdesc}{getResult}{timeout=None,timeoutAccept=True}
 Get the results from the input dialog.

        This fuction is used to present a modal dialog to the user (i.e. a
        dialog that must be ended before the user can continue with the
        program. The dialog is shown and user interaction is processed.
        The user ends the interaction either by accepting the data (e.g. by
        pressing the OK button or the ENTER key) or by rejecting them (CANCEL
        button or ESC key).
        On accept, a dictionary with all the fields and their values is
        returned. On reject, an empty dictionary is returned.
        
        If a timeout (in seconds) is given, a timer will be started and if no
        user input is detected during this period, the input dialog returns
        with the default values set.
        A value 0 will timeout immediately, a negative value will never timeout.
        The default is to use the global variable input_timeout.

        This function also sets the exit mode, so that the caller can test how
        the dialog was ended.
        self.accepted == TRUE/FALSE
        self.timedOut == TRUE/FALSE
        
\end{funcdesc}

\subsection{TableModel --- A table model that represent data as a two-dimensional array of items.}
    data is any tabular data organized in a fixed number of rows and colums.
    This means that an item at row i and column j can be addressed as
    data[i][j].
    Optional lists of column and row headers can be specified.
    

\begin{classdesc}{TableModel}{data,chead=None,rhead=None,parent=None,*args}

\end{classdesc}

TableModel instances have the following methods:

\begin{funcdesc}{rowCount}{parent=None}

\end{funcdesc}

\begin{funcdesc}{columnCount}{parent=None}

\end{funcdesc}

\begin{funcdesc}{data}{index,role}

\end{funcdesc}

\begin{funcdesc}{headerData}{col,orientation,role}

\end{funcdesc}

\begin{funcdesc}{insertRows}{row=None,count=None}

\end{funcdesc}

\begin{funcdesc}{removeRows}{row=None,count=None}

\end{funcdesc}

\subsection{Table --- A dialog widget to show two-dimensional arrays of items.}


\begin{classdesc}{Table}{data,chead=None,rhead=None,caption=None,parent=None,actions=[('OK',)],default='OK'}
Create the Table dialog.
        
        data is a 2-D array of items, mith nrow rows and ncol columns.
        chead is an optional list of ncol column headers.
        rhead is an optional list of nrow row headers.
        
\end{classdesc}

\subsection{TableDialog --- A dialog widget to show two-dimensional arrays of items.}


\begin{classdesc}{TableDialog}{items,caption=None,parent=None,tab=False}
Create the Table dialog.
        
        If tab = False, a dialog with one table is created and items
        should be a list [table_header,table_data].
        If tab = True, a dialog with multiple pages is created and items
        should be a list of pages [page_header,table_header,table_data].
        
\end{classdesc}

\subsection{ButtonBox --- }


\begin{classdesc}{ButtonBox}{name,choices,funcs,*args}

\end{classdesc}

ButtonBox instances have the following methods:

\begin{funcdesc}{setText}{text,index=0}

\end{funcdesc}

\begin{funcdesc}{setIcon}{icon,index=0}

\end{funcdesc}

\begin{funcdesc}{__str__}{}

\end{funcdesc}

\subsection{ComboBox --- }


\begin{classdesc}{ComboBox}{name,choices,func,*args}

\end{classdesc}

ComboBox instances have the following methods:

\begin{funcdesc}{setIndex}{i}

\end{funcdesc}

\subsection{BaseMenu --- A general menu class.}
    This class is not intended for direct use, but through subclasses.
    Subclasses should implement at least the following methods:
      addSeparator()              insertSeperator(before)
      addAction(text,action)      insertAction(before,text,action)
      addMenu(text,menu)          insertMenu(before,text,menu)
      
    QtGui.Menu and QtGui.MenuBar provide these methods.
    

\begin{classdesc}{BaseMenu}{title='AMenu',parent=None,before=None,items=None}
Create a menu.

        This is a hierarchical menu that keeps a list of its item
        names and actions.
        
\end{classdesc}

BaseMenu instances have the following methods:

\begin{funcdesc}{item}{text}
Get the menu item with given normalized text.

        Text normalization removes all '\&' characters and
        converts to lower case.
        
\end{funcdesc}

\begin{funcdesc}{itemAction}{item}
Return the action corresponding to item.

        item is either one of the menu's item texts, or one of its
        values. This method guarantees that the return value is either the
        corresponding Action, or None.
        
\end{funcdesc}

\begin{funcdesc}{insert_sep}{before=None}
Create and insert a separator
\end{funcdesc}

\begin{funcdesc}{insert_menu}{menu,before=None}
Insert an existing menu.
\end{funcdesc}

\begin{funcdesc}{insert_action}{action,before=None}
Insert an action.
\end{funcdesc}

\begin{funcdesc}{create_insert_action}{str,val,before=None}
Create and insert an action.
\end{funcdesc}

\begin{funcdesc}{insertItems}{items,before=None}
Insert a list of items in the menu.
        
        Each item is a tuple of two to five elements:
           Text, Action, [ Icon,  ShortCut, ToolTip ].

        Item text is the text that will be displayed in the menu.
        It will be stored in a normalized way: all lower case and with
        '\&' removed.

        Action can be any of the following:
          - a Python function or instance method : it will be called when the
            item is selected,
          - a string with the name of a function/method,
          - a list of Menu Items: a popup Menu will be created that will appear
            when the item is selected,
          - None : this will create a separator item with no action.

        Icon is the name of one of the icons in the installed icondir.
        ShortCut is an optional key combination to select the item.
        Tooltip is a popup help string.

        If before is given, it specifies the text OR the action of one of the
        items in the menu: the new items will be inserted before that one.
        
\end{funcdesc}

\subsection{Menu --- A popup/pulldown menu.}


\begin{classdesc}{Menu}{title='UserMenu',parent=None,before=None,items=None}
Create a popup/pulldown menu.

        If parent==None, the menu is a standalone popup menu.
        If parent is given, the menu will be inserted in the parent menu.
        If parent==GD.GUI, the menu is inserted in the main menu bar.
        
        If insert == True, the menu will be inserted in the main menubar
        before the item specified by before.
        If before is None or not the normalized text of an item of the
        main menu, the new menu will be inserted at the end.
        Calling the close() function of an inserted menu will remove it
        from the main menu.

        If insert == False, the created menu will be an independent dialog
        and the user will have to process it explicitely.
        
\end{classdesc}

Menu instances have the following methods:

\begin{funcdesc}{process}{}

\end{funcdesc}

\begin{funcdesc}{remove}{}
Remove this menu from its parent.
\end{funcdesc}

\subsection{MenuBar --- A menu bar allowing easy menu creation.}


\begin{classdesc}{MenuBar}{}
Create the menubar.
\end{classdesc}

\subsection{DAction --- A DAction is a QAction that emits a signal with a string parameter.}
    When triggered, this action sends a signal (default 'Clicked') with a
    custom string as parameter. The connected slot can then act depending
    on this parameter.
    

\begin{classdesc}{DAction}{name,icon=None,data=None,signal=None}
Create a new DAction with name, icon and string data.

        If the DAction is used in a menu, a name is sufficient. For use
        in a toolbar, you will probably want to specify an icon.
        When the action is triggered, the data is sent as a parameter to
        the SLOT function connected with the 'Clicked' signal.
        If no data is specified, the name is used as data. 
        
        See the views.py module for an example.
        
\end{classdesc}

DAction instances have the following methods:

\begin{funcdesc}{activated}{}

\end{funcdesc}

\subsection{ActionList --- Menu and toolbar with named actions.}
    An action list is a list of strings, each connected to some action.
    The actions can be presented in a menu and/or a toolbar.
    On activating one of the menu or toolbar buttons, a given signal is
    emitted with the button string as parameter. A fixed function can be
    connected to this signal to act dependent on the string value.
    

\begin{classdesc}{ActionList}{actions=[],function=None,menu=None,toolbar=None,icons=None}
Create an new action list, empty by default.

        A list of strings can be passed to initialize the actions.
        If a menu and/or toolbar are passed, a button is added to them
        for each string in the action list.
        If a function is passed, it will be called with the string as
        parameter when the item is triggered.

        If no icon names are specified, they are taken equal to the
        action names. Icons will be taken from the installed icon directory.
        If you want to specify other icons, use the add() method.
        
\end{classdesc}

ActionList instances have the following methods:

\begin{funcdesc}{add}{name,icon=None}
Add a new name to the actions list and create a matching DAction.

        If the actions list has an associated menu or toolbar,
        a matching button will be inserted in each of these.
        If an icon is specified, it will be used on the menu and toolbar.
        The icon is either a filename or a QIcon object. 
        
\end{funcdesc}

\begin{funcdesc}{names}{}
Return an ordered list of names of the action items.
\end{funcdesc}

\subsection{Functions defined in module \module{widgets}}

\begin{funcdesc}{selectFont}{}
Ask the user to select a font.

    A font selection dialog widget is displayed and the user is requested
    to select a font.
    Returns a font if the user exited the dialog with the OK button.
    Returns None if the user clicked CANCEL.
    
\end{funcdesc}

\begin{funcdesc}{getColor}{col=None,caption=None}
Create a color selection dialog and return the selected color.

    col is the initial selection.
    If a valid color is selected, its string name is returned, usually as
    a hex \#RRGGBB string. If the dialog is canceled, None is returned.
    
\end{funcdesc}

\begin{funcdesc}{dialogButtons}{dialog,actions,default}
Create a set of dialog buttons

    dia is a dialog widget
    actions is a list of tuples (name,) or (name,function).
    If a function is specified, it will be executed on pressing the button.
    If no function is specified, and name is one of 'ok' or 'cancel' (case
    does not matter), the button will be bound to the dialog's 'accept'
    or 'reject' slot.
    default is the name of the action to set as the default.
    
\end{funcdesc}

\begin{funcdesc}{messageBox}{message,level='info',choices=['OK'],timeout=None}
Display a message box and wait for user response.

    The message box displays a text, an icon depending on the level
    (either 'about', 'info', 'warning' or 'error') and 1-3 buttons
    with the specified action text. The 'about' level has no buttons.

    The function returns the text of the button that was clicked or
    an empty string is ESC was hit.
    
\end{funcdesc}

\begin{funcdesc}{textBox}{text,choices=['OK']}
Display a text and wait for user response.

    Possible choices are 'OK' and 'CANCEL'.
    The function returns True if the OK button was clicked or 'ENTER'
    was pressed, False if the 'CANCEL' button was pressed or ESC was pressed.
    
\end{funcdesc}

\begin{funcdesc}{normalize}{s}
Normalize a string.

    Text normalization removes all '\&' characters and converts to lower case.
    
\end{funcdesc}

%%% Local Variables: 
%%% mode: latex
%%% TeX-master: "pyformex"
%%% End:

