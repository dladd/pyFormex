% pyformex manual --- faq
% $Id$
% (C) B.Verhegghe

\chapter{pyFormex FAQ 'n TRICKS}
\label{cha:faq}
\def\faq#1{\item\textbf{#1}\par}
\def\trick#1{\item\textbf{#1}\par}

This chapter answers some frequently asked questions about \pyformex and present some nice tips to solve common problems. If you have some question that you want answered, or want to present a original solution to some problem, feel free to communicate it to us\footnote{By preference via the forums on the \pyformex web site}, and we'll probably include it in the next version.  

\section{FAQ}
\label{Sec:faq}
\begin{enumerate}

\faq{How was the pyFormex logo created?}
%---------------------------------------
With the Gimp, using the following command sequence: 
\begin{verbatim}
Xtra -> Script-Fu -> Logos -> Alien-neon
Font Size: 150 
Font: Blippo-Heavy
Glow Color: 0xFF3366
Background Color: 0x000000
Width of Bands: 2
Width of Gaps: 2
Number of Bands: 7
Fade Away: Yes
\end{verbatim}
Then switch off the background layer and save the image in PNG format.
Export the image with 'Save Background Color' switched off!



\faq{Why is pyFormex written in Python?}
%---------------------------------------
Because
\begin{itemize}
\item it is very easy to learn (See www.python.org)
\item it is extremely powerful (More on www.python.org)
\end{itemize}

Being a scripting language without the need for variable declaration, it allows for quick program development.
On the other hand, Python provides numerous interfaces with established compiled libraries, so it can be surprisingly fast.


\faq{Is an interpreted language like Python fast enough with large data models?}
%-----------------------------------------------------------------------------
See the question above


\end{enumerate}


\section{TRICKS}
\label{sec:tricks}

\begin{enumerate}
\trick{Set the directory where a script is found as the current working directory}
%----------------------------------------------------------------------
Start your script with the following code snippet:
\begin{verbatim}
import os
os.chdir(os.path.dirname(GD.cfg['curfile']))
\end{verbatim}
or the equivalent shortcut:
\begin{verbatim}
workHere()
\end{verbatim}

\trick{Import modules from your own script directories}
%------------------------------------------------------
In order for Python to find the modules in non-standard locations, you should add the directory path of the module to the \verb|sys.path| variable. 

A common example is a script that wants to import modules from the same directory where it is located. In that case you can just add the following two lines to the start of your script.
\begin{verbatim}
import os,sys
sys.path.insert(os.dirname(GD.cfg['curfile']))
\end{verbatim}

\trick{Automatically load plugin menus on startup}
%--------------------------------------------------------
Plugin menus can be loaded automatically on \pyformex startup, by adding a line to the \code{gui} section of your configuration file (\verb|~/.pyformexrc|).
\begin{verbatim}
[gui]
plugins = ['surface_menu', 'formex_menu']
\end{verbatim}

\trick{Automatically execute your own scripts on startup}
%-------------------------------------------------
If you create your own pugin menus for \pyformex, you cannot autoload them like the regular plugin menus from the distribution, because they are not in the plugin directory of the installation.\footnote{Do not be tempted to put your own files under the installation directory (even if you can acquire the permissions to do so), because on removal or reinstall your files might be deleted!}
You can however automatically execute your own scripts by adding their full path names in the \code{autorun} variable of your configuration file.
\begin{verbatim}
autorun = '/home/user/myscripts/startup/'
\end{verbatim}
This script will then be run when the \pyformex GUI starts up. You can even specify a list of scripts, which will be executed in order.
The autorun scripts are executed as any other \pyformex script, before any scripts specified on the command line, and before giving the input focus to the user.

\trick{Create a movie from a sequence of recorded images}
%--------------------------------------------------
The multisave option allows you to easily record a series of images while working with \pyformex. You may want to turn this sequence into a movie afterwards. THis can be done with the \Code{mencoder} and/or \Code{ffmpeg} programs. The internet provides comprehensive information on how to use these video encoders. 

If you are looking for a quick answer, however, here are some of the commands we have often used to create movies.
\begin{itemize}
\item Create MNPG movies from PNG
To keep the quality of the PNG images in your movie, you should not encode them into a compressed format like MPEG. You can use the MPNG codec instead. Beware though that uncompressed encodings may lead to huge video files. Also, the MNPG is (though freely available), not installed by default on Windows machines.

Suppose you have images in files \Code{image-000.png}, \Code{image-001.png}, \dots.
First, you should get the size of the images (they all should have the same size). The command 
\begin{verbatim}
   file image*.png
\end{verbatim}
 will tell you the size. Then create movie with the command
\begin{verbatim}
  mencoder mf://image-*.png -mf w=796:h=516:fps=5:type=png -ovc copy -oac copy -o movie1.avi
\end{verbatim}
Fill in the correct width(w) and height(h) of the images, and set the frame rate(fps). The result will be a movie \Code{movie1.avi}.


\item Create a movie from (compressed) JPEG images.
Because the compressed format saves a lot of space, this will be the prefered format if you have lots of image files. The quality of the compressed image movie will suffer somewhat, though.
\begin{verbatim}
  ffmpeg -r 5 -b 800 -i image-%03d.jpg movie.mp4
\end{verbatim}


\trick{Install the gl2ps extension}
%------------------------------------
Saving images in EPS format is done through the gl2ps library, which can be accessed from Python using the wrappers from python-gl2ps-1.1.2.tar.gz.

You need to have the OpenGL header files installed in order to do this. (On Debian: apt-get install libgl1-mesa-dev.)

\begin{verbatim}
tar xvzf python-gl2ps-1.1.2.tar.gz
cd python-gl2ps-1.1.2
su root
python setup.py install
\end{verbatim}

\end{itemize}
\end{enumerate}


%%% Local Variables: 
%%% mode: latex
%%% TeX-master: "manual"
%%% End: 
