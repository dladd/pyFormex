% pyformex manual --- faq
% $Id$
% (C) B.Verhegghe

\chapter{pyFormex FAQ 'n TRICKS}
\label{cha:faq}

This chapter answers some frequently asked questions about \pyformex and present some nice tips to solve common problems. If you have some question that you want answered, or want to present a original solution to some problem, feel free to communicate it to us\footnote{By preference via the forums on the \pyformex web site}, and we'll probably include it in the next version.  

\section{FAQ}
\label{Sec:faq}
\begin{enumerate}
\item \emph{How was the pyFormex logo created?}

With the Gimp, using the following command sequence: 
\begin{verbatim}
Xtra -> Script-Fu -> Logos -> Alien-neon
Font Size: 150 
Font: Blippo-Heavy
Glow Color: 0xFF3366
Background Color: 0x000000
Width of Bands: 2
Width of Gaps: 2
Number of Bands: 7
Fade Away: Yes
\end{verbatim}
Then switch off the background layer and save the image in PNG format.
Export the image with 'Save Background Color' switched off!



\item \emph{Why is pyFormex written in Python? Isn't it slow?}

Because
\begin{itemize}
\item it is very easy to learn (See www.python.org)
\item it is extremely powerful (More on www.python.org)
\end{itemize}

Being a scripting language without the need for variable declaration, it allows for quick program development.
On the other hand, Python provides numerous interfaces with established compiled libraries, so it can be surprisingly fast.


\end{enumerate}


\section{TRICKS}
\label{sec:tricks}

\begin{enumerate}
\item Set the directory where a script is found as the current working directory:
Start your script with the following code snippet:
\begin{verbatim}
import os
os.chdir(os.path.dirname(GD.cfg['curfile']))
\end{verbatim}

\item Automatically load plugin menus on startup:
Plugin menus can be loaded automatically on pyFormex startup, by adding a line to the \code{gui} section of your configuration file (\verb|~/.pyformexrc|).
\begin{verbatim}
[gui]
plugins = ['surface_menu', 'formex_menu']
\end{verbatim}

\end{enumerate}


%%% Local Variables: 
%%% mode: latex
%%% TeX-master: "manual"
%%% End: 
