
%%%%%%%%%%%%%%%%%%%%%%%%%%%%%%%%%%%%%%%%%%%%%%%%%%%%%%%%%%%%%%%%%%%%%%%%%%%
\chapter{\pyformex --- reference manual}
{\label{cha:reference}

\section{formex --- the base module}
{\label{sec:formex}

This module contains all the basic functionality for creating, structuring and transforming sets of coordinates. All the definitions in this module are available in your scripts without the need to import the module.

\begin{classdesc}{Formex}{data=[[[]]],prop=None}
A class to hold a structured set of coordinates. A \class{Formex} is a three dimensional array of float values. The array has a shape \Code{(nelems,nplex,3)}. Each slice \Code{[i,j]} of the array contains the three coordinates of a point in space. Each slice \Code{[i]} of the array contains a connected set of \var{nplex} points: we will refer to it as an \emph{element}. The number of points in an element is also called the \emph{plexitude} of the element.

It is up to the user on how to interprete the grouping of some points into an element. An element with two points  will usually represent a line segment between its two points. But these could just as well be the two opposite corners of a rectangle.
A element with plexitude 3 (in short: plex-3 element) could be interpreted as a triangle or as a polygonal or curved line. And if it is a triangle, it could be either the circumference of the triangle or the part of the plane inside that circumference. As far as the Formex class concerns, each element is just a set of points. 

All elements in a \class{Formex} must have the same number of points, but you can construct \class{Formex} instances with any (positive) number of nodes per element. A plex-1 \class{Formex} is just a collection of unconnected nodes (each element is a single point).

One way of attaching other data to the \class{Formex}, is by the use of the 'property' attribute. The property is an array holding one integer value for each of the elements of the Formex. The use of this property value is completely defined by the user. It could be a code for the type of element, or for the color to draw this element with. Most often it will be used as an index into some other (possibly complex) data structure holding all the characteristics of that element. 

By including this property index into the Formex class, we make sure that when new elements are constructed from existing ones, the element properties are automatically propagated.

\end{classdesc}

\subsection{Formex class members}

\begin{memberdesc}  [array]{f}
A three dimensional array of float values. The array always has a shape with three components (nelems,nplex,3), even if the Formex is empty. An empty Formex has shape[0] == 0.
\end{memberdesc}

\begin{memberdesc}  [array]{p}
Either an integer array with shape (nelems,), or None. If not None, an integer value is attributed to each element of the Formex. There is no provision to attribute different values to the separate nodes of an element. If you need such functionality, use the \var{p} array as a pointer into a data structure that has different values per node.

The \var{p} is called property number or property for short. If it is not None, it will take part in the Formex transformations and its values will propagate to all copies created from the Formex elements.
\end{memberdesc}

\subsection{Basic access methods}

\begin{methoddesc}{__getitem__}{i}
This is equivalent to \Code{self.f.__getitem__(i)}. It allows to access the data in the coordinate array \var{f} of the Formex with all the index methods of numpy. The result is an float array or a single float. Thus: \Code{F[1]} returns the second element of \var{F}, \Code{F[1,0]} the first point of that element and \Code{F[1,0,2]} the z-coordinate of that point. \Code{F[:,1]} is an array with the second point of all elements. \Code{F[:,:,1]} is the y-coordinate of all points of all elements in the Formex.
\end{methoddesc}

\begin{methoddesc}{__setitem__}{i,val}
This is equivalent to \Code{self.f.__getitem__(i)}. It allows to change individual elements, points or coordinates using the item selection syntax. Thus: \Code{F[1:5,1,2] = 1.0} sets the z-coordinate of the second points of the elements 1, 2, 3 and 4 to the value 1.0.
\end{methoddesc}

\begin{methoddesc}{element}{i}
Returns the element \var{i}. \Code{F.element[i]} is currently equivalent with \Code{F[i]}.
\end{methoddesc}

\begin{methoddesc}{point}{i,j}
Returns the point \var{j} of the element \var{i}. \Code{F.point[i,j]} is equivalent with \Code{F[i,j]}.
\end{methoddesc}

\begin{methoddesc}{coord}{i,j,k}
Returns the coordinate \var{k} of the point \var{j} of the element \var{i}. \Code{F.coord[i,j,k]} is equivalent with \Code{F[i,j,k]}.
\end{methoddesc}


\subsection{Methods returning information}

\begin{methoddesc}{nelems}{}
Returns the number of elements in the Formex.
\end{methoddesc}

\begin{methoddesc}{nplex}{}
Returns the number of points in each element.
\end{methoddesc}
    
\begin{methoddesc}{ndim}{self}
Returns the number of dimensions. This is the number of coordinates for each point. 

In the current implementation this is always 3, though you can define 2D Formices by given only two coordinates: the third will automatically be set to zero.
\end{methoddesc}

\begin{methoddesc}{npoints}{}
Return the number of points in the Formex.

This is the product of the \var{nelems()} and  \var{nplex()}.
\end{methoddesc}
    
\begin{methoddesc}{shape}{}
Return the shape of the Formex.

The shape of a Formex is the shape of its data array,
i.e. a tuple (nelems, nplex, ndim).
\end{methoddesc}

\begin{methoddesc}{data}{}
  Return the Formex as a numpy array.
\end{methoddesc}
\begin{methoddesc}{x}{}
  Return the x-plane.
\end{methoddesc}
\begin{methoddesc}{y}{}
  Return the y-plane.
\end{methoddesc}
\begin{methoddesc}{z}{}
  Return the z-plane.
\end{methoddesc}

\begin{methoddesc}{prop}{}
Return the properties as a numpy array, or None if the Formex has no properties.
\end{methoddesc}

\begin{methoddesc}{maxprop}{}
Return the highest property used, or None if the Formex has no properties.
\end{methoddesc}

\begin{methoddesc}{propSet}{}
Return a list with the unique property values on this Formex, or None if the Formex has no properties.
\end{methoddesc}


\begin{methoddesc}{bbox}{}
Return the bounding box of the Formex.

The bounding box is the smallest rectangular volume in global coordinates, such that no points of the Formex are outside the box. It is returned as a [2,3] array: the first row holds the minimal coordinates and the second one the maximal.
\end{methoddesc}

\begin{methoddesc}{center}{}
Return the center of the Formex. This is the center of its \function{bbox()}.
\end{methoddesc}

\begin{methoddesc}{sizes}{}
Returns an array with shape (3,) holding the length of the bbox along the 3 axes.
\end{methoddesc}

\begin{methoddesc}{size}{}
Return the size of the Formex. This is defined as the length of the diagonal of the bbox().
\end{methoddesc}

\begin{methoddesc}{bsphere}{}
Return the diameter of the bounding sphere of the Formex.

The bounding sphere is the smallest sphere with center in the \function{center()} of the Formex, and such that no points of the Formex are lying outside the sphere. It is not necessarily the smallest sphere surrounding all points of the Formex.
\end{methoddesc}

\begin{methoddesc}{feModel}{nodesperbox=1,repeat=True,rtol=1.e-5,atol=1.e-5}
Return a tuple of nodal coordinates and element connectivity.

A tuple of two arrays is returned. The first is a float array with the coordinates of the unique nodes of the Formex. The second is an integer array with the node numbers connected by each element. The elements come in the same order as they are in the Formex, but the order of the nodes is unspecified. By the way, the reverse operation of
\Code{coords,elems = feModel(F)} is accomplished by \Code{F = Formex(coords[elems])}. There is a (very small) probability that two very close nodes are not equivalenced by this procedure. Use it multiple times with different parameters to check.

\var{rtol} and \var{atol} are the relative, resp. absolute tolerances used to decide whether any nodal coordinates are considered to be equal. 
\end{methoddesc}


\subsection{Methods returning string representations}

\begin{methoddesc}{point2str}{point}
Return a string representation of a point. The string holds the three coordinates, separated by a comma. \classmethod
\end{methoddesc}

\begin{methoddesc}{elem2str}{elem}
Return a string representation of an element. The string contains the string representations of all the element's nodes, separated by a semicolon. \classmethod
\end{methoddesc}
    
\begin{methoddesc}{asFormex}{}
Return a string representation of a Formex.

Coordinates are separated by commas, points are separated by semicolons and grouped between brackets, elements are separated by commas and grouped between braces. Thus a Formex \Code{F = Formex([[[1,0],[0,1]],[[0,1],[1,2]]])} is formatted as '\{[1.0,0.0,0.0; 0.0,1.0,0.0], [0.0,1.0,0.0; 1.0,2.0,0.0]\}'.
\end{methoddesc}

\begin{methoddesc}{asFormexWithProp}{}
Return string representation as Formex with properties. The string representation as done by asFormex() is followed by the words "with prop" and a list of the properties.
\end{methoddesc}
              
\begin{methoddesc}{asArray}{}
Return a string representation of the Formex as a numpy array.
\end{methoddesc}

\begin{methoddesc}{setPrintFunction}{func}
This sets how a Formex will be formatted by the print statement or by a \Code{"\%s"} formatting string. \var{func} can be any of the above functions \var{asFormex}, \var{asFormexWithProp} or \var{asArray}, or a user-defined function. 

\classmethod
Use it as follows: \Code{Formex.setPrintFunction(Formex.asArray)}.
\end{methoddesc}


\begin{methoddesc}{fprint}{fmt="\%10.3e \%10.3e \%10.3e"}
Prints all the points of the formex with the specified format. The format should hold three formatting codes, for the three coordinates of the point. 
\end{methoddesc}


\subsection{Methods changing the instance data}
These are the only methods that change the data of the Formex object. All other transformation methodes return and operate on copies, leaving the original object unchanged.

\begin{methoddesc}{setProp}{p}
Create a property array for the Formex.

A property array is a rank-1 integer array with dimension equal to the number of elements in the Formex (first dimension of data). You can specify a single value or a list/array of integer values.

If the number of passed values is less than the number of elements, they wil be repeated. If you give more, they will be ignored. 

Specifying a value \Code{None} results in a Formex without properties.
\end{methoddesc}

\begin{methoddesc}{append}{F}
Appends all the elements of a Formex F to the current one. Use the \var{__add__} function or the \Code{+} operator to concatenate two Formices without changing either of the onjects.

Only Formices having the same plexitude as the current one can be appended.
If one of the Formices has properties and the other not, the elements with missing properties will be assigned property 0.
\end{methoddesc}


\subsection{Methods returning copies}

\begin{methoddesc}{copy}{}
Return a deep copy of itself.
\end{methoddesc}


\begin{methoddesc}{__add__}{other}
Return a Formex with all elements of self and other. This allows the user to write simple expressions as F+G to concatenate the Formices F and G. As with the append() method, both Formices should have the same plexitude.
\end{methoddesc}


\begin{methoddesc}{concatenate}{Flist}
Return the concatenation of all formices in Flist. All formices should have the same plexitude. \classmethod 
\end{methoddesc}


\begin{methoddesc}{select}{idx}
Return a Formex which holds only elements with numbers in \var{idx}.
\var{idx} can be a single element number or a list of numbers.
\end{methoddesc}

\begin{methoddesc}{selectNodes}{idx}
Return a Formex which holds only some nodes of the parent. \var{idx} is a list of node numbers to select.\\
Thus, if F is a grade 3 Formex representing triangles, the sides of the triangles are given by\\
\verb? F.selectNodes([0,1]) + F.selectNodes([1,2]) + F.selectNodes([2,0]) ?\\
The returned Formex inherits the property of its parent.
\end{methoddesc}

\begin{methoddesc}{nodes}{}
Return a Formex containing only the nodes.

This is obviously a Formex with plexitude 1. It holds the same data as the original Formex, but in another shape: the number of nodes per element is 1, and the number of elements is equal to the total number of nodes. The properties are not copied over, since they will usually not make any sense.
\end{methoddesc}

\begin{methoddesc}{remove}{F}
Return a Formex where the elements in \var{F} have been removed.

This is also the subtraction of the current Formex with \var{F}. Elements are only removed if they have the same nodes in the same order. This is a slow operation: for large structures, you should avoid it where possible.
\end{methoddesc}

\begin{methoddesc}{withProp}{val}
Return a Formex which holds only the elements with property val.

If the Formex has no properties, a copy is returned.
The returned Formex is always without properties.
\end{methoddesc}

\begin{methoddesc}{elbbox}{}
Return a Formex where each element is replaced by its bbox.

The returned Formex has two points for each element: two corners
of the bbox.
\end{methoddesc}

\begin{methoddesc}{unique}{self,rtol=1.e-4,atol=1.e-6}
Return a Formex which holds only the unique elements.

Two elements are considered equal when all its nodal coordinates
are close. Two values are close if they are both small compared to atol
or their difference divided by the second value is small compared to
rtol.
Two elements are not considered equal if one's elements are a
permutation of the other's.

Warning: this operation is slow when performed on large Formices.
\end{methoddesc}

\begin{methoddesc}{reverseElements}{}
Return a Formex where all elements have been reversed.

Reversing an element means reversing the order of its points.
\end{methoddesc}


\subsection{Clipping methods}
These methods can be use to make a selection of elements based on their nodal coordinates. The heart is the function
\begin{methoddesc}{test}{nodes='all',dir=0,min=None,max=None}
Flag elements having nodal coordinates between min and max.

This function is very convenient in clipping a Formex in one of
the coordinate directions. It returns a 1D integer array flagging
(with a value 1) the elements having nodal coordinates in the
specified range.
Use \Code{where(result)} to get a list of element numbers passing the test.
Or directly use the \Code{clip()} or \Code{cclip()} methods to create the clipped Formex.

xmin,xmax are there minimum and maximum values required for the
coordinates in direction dir (default is the x or 0 direction).
nodes specifies which nodes are taken into account in the comparisons.
It should be one of the following:
\begin{itemize}
\item a single (integer) node number (< the number of nodes)
\item a list of node numbers
\item one of the special strings: 'all', 'any', 'none'
\end{itemize}
The default ('all') will flag all the elements that have all their
nodes between the planes x=min and x=max, i.e. the elements that
fall completely between these planes. One of the two clipping planes
may be left unspecified.
\end{methoddesc}

If you want to have a list of the element numbers that satisfy the specified conditions, you can use numpy's where function on the result. Thus \Code{where(F.where(min=1.0))} returns a list with all elements lying right of the plane \Code{x=1.0}.

\begin{methoddesc}{clip}{t}
Returns a Formex with all the elements where t>0.

t should be a 1-D integer array with length equal to the number
of elements of the formex.
The resulting Formex will contain all elements where t > 0.
This is a convenience function for the user, equivalent to
\Code{F.select(t>0)}.
\end{methoddesc}

\begin{methoddesc}{cclip}{t}
This is the complement of clip, returning a Formex where t<=0.
\end{methoddesc}



\subsection{Affine transformations}

\begin{methoddesc}{scale}{scale}
Returns a copy scaled with \Code{scale[i]} in direction \Code{i}.

The \var{scale} should be a list of 3 numbers, or a single number. In the latter case, the scaling is homothetic.
\end{methoddesc}

\begin{methoddesc}{translate}{dir,distance=None}
Returns a copy translated over \var{distance} in direction \var{dir}.

\var{dir} is either an axis number (0,1,2) or a direction vector.

If a distance is given, the translation is over the specified
distance in the specified direction.
If no distance is given, and \var{dir} is specified as an axis number,
translation is over a distance 1.
If no distance is given, and \var{dir} is specified as a vector, translation
is over the specified vector.

Thus, the following are all equivalent:
\Code{F.translate(1)};
\Code{F.translate(1,1)};
\Code{F.translate([0,1,0])};
\Code{F.translate([0,2,0],1)}
\end{methoddesc}

\begin{methoddesc}{rotate}{angle,axis=2}
Return a copy rotated over \var{angle} around \var{axis}.

The angle is specified in degrees.
The axis is either one of 0,1,2 designating one of the global axes,
or a 3-component vector specifying an axis through the origin.
If no axis is specified, rotation is around the 2(z)-axis. This is
convenient for working on 2D-structures.

Positive angles rotate clockwise when looking in the positive direction of the axis.

As a convenience, the user may also specify a 3x3 rotation matrix as argument.
In that case rotate(mat) is equivalent to affine(mat).
\end{methoddesc}


\begin{methoddesc}{shear}{dir,dir1,skew}
Returns a copy skewed in the direction \var{dir} of plane \var{(dir,dir1)}.

The coordinate \var{dir} is replaced with \Code{(dir + skew * dir1)}.
\end{methoddesc}

\begin{methoddesc}{reflect}{dir,pos=0}
Returns a Formex mirrored in direction \var{dir} against plane at \var{pos}.

Default position of the plane is through the origin.
\end{methoddesc}

\begin{methoddesc}{affine}{mat,vec=None}
Returns a general affine transform of the Formex.

The returned Formex has coordinates given by \Code{mat * xorig + vec}, where \var{mat} is a 3x3 matrix and \var{vec} a length 3 list.
\end{methoddesc}


\subsection{Non-affine transformations}

\begin{methoddesc}{cylindrical}{dir=[0,1,2],scale=[1.,1.,1.]}
Converts from cylindrical to cartesian after scaling.

\var{dir} specifies which coordinates are interpreted as resp. distance(r), angle(theta) and height(z). Default order is [r,theta,z].\\
\var{scale} will scale the coordinate values prior to the transformation. (scale is given in order r,theta,z). The resulting angle is interpreted in degrees.
\end{methoddesc}


\begin{methoddesc}{toCylindrical}{dir=[0,1,2]}
Converts from cartesian to cylindrical coordinates.

\var{dir} specifies which coordinates axes are parallel to respectively the cylindrical axes distance(r), angle(theta) and height(z). Default order is [x,y,z]. The angle value is given in degrees.
\end{methoddesc}

\begin{methoddesc}{spherical}{dir=[0,1,2],scale=[1.,1.,1.],colat=False}
Converts from spherical to cartesian after scaling.

\var{dir} specifies which coordinates are interpreted as longitude(theta), latitude(phi) and distance(r).\\
\var{scale} will scale the coordinate values prior to the transformation.\\
Angles are then interpreted in degrees.\\
Latitude, i.e. the elevation angle, is measured from equator in
direction of north pole(90). South pole is -90.
If colat=True, the third coordinate is the colatitude (90-lat) instead.
That choice may facilitate the creation of spherical domes.
\end{methoddesc}

\begin{methoddesc}{toSpherical}{dir=[0,1,2]}
Converts from cartesian to spherical coordinates.

\var{dir} specifies which coordinates axes are parallel to respectively the spherical axes distance(r), longitude(theta) and colatitude(phi). Colatitude is 90 degrees - latitude, i.e. the elevation angle measured from north pole(0) to south pole(180). Default order is [0,1,2], thus the equator plane is the (x,y)-plane. The returned angle values are given in degrees.
\end{methoddesc}

\begin{methoddesc}{bump1}{dir,a,func,dist}
Return a Formex with a one-dimensional bump.

\var{dir} specifies the axis of the modified coordinates.\\
\var{a} is the point that forces the bumping.\\
\var{dist} specifies the direction in which the distance is measured.\\
\var{func} is a function that calculates the bump intensity from distance. \Code{func(0)} should be different from 0.
\end{methoddesc}

\begin{methoddesc}{bump2}{dir,a,func}
Return a Formex with a two-dimensional bump.

\var{dir} specifies the axis of the modified coordinates.\\
\var{a} is the point that forces the bumping.\\
\var{func} is a function that calculates the bump intensity from distance. \Code{func(0)} should be different from 0.
\end{methoddesc}

\begin{methoddesc}{bump}{dir,a,func,dist=None}
Return a Formex with a bump.

A bump is a modification of a set of coordinates by a non-matching point. It can produce various effects, but one of the most common uses is to force a surface to be indented by some point.
        
\var{dir} specifies the axis of the modified coordinates.\\
\var{a} is the point that forces the bumping.\\
\var{func} is a function that calculates the bump intensity from distance. \Code{func(0)} should be different from 0.\\
\var{dist} is the direction in which the distance is measured : this can be one of the axes, or a list of one or more axes. If only 1 axis is specified, the effect is like function \function{bump1()}. If 2 axes are specified, the effect is like \function{bump2}. This function can take 3 axes however. Default value is the set of 3 axes minus the direction of modification. This function is then equivalent to \function{bump2()}.
\end{methoddesc}

\begin{methoddesc}{map}{func}
Return a Formex mapped by a 3-D function.

This is one of the versatile mapping functions.\\
\var{func} is a numerical function which takes three arguments and produces a list of three output values. The coordinates [x,y,z] will be replaced by func(x,y,z). The function must be applicable to arrays, so it should only include numerical operations and functions understood by the numpy module.

This method is one of several mapping methods. See also \function{map1()} and \function{mapd()}.\\
Example: \Code{E.map(lambda x,y,z: [2*x,3*y,4*z])} is equivalent with \Code{E.scale([2,3,4])}.
\end{methoddesc}

\begin{methoddesc}{map1}{dir,func}
Return a Formex where coordinate \var{i} is mapped by a 1-D function.

\var{func} is a numerical function which takes one argument and produces one result. The coordinate \var{dir} will be replaced by func(coord[dir]). The function must be applicable on arrays, so it should only include numerical operations and functions understood by the numpy module. This method is one of several mapping methods. See also \function{map()} and \function{mapd()}.
\end{methoddesc}

\begin{methoddesc}{mapd}{dir,func,point,dist=None}
Maps one coordinate by a function of the distance to a point.

\var{func} is a numerical function which takes one argument and produces one result. The coordinate \var{dir} will be replaced by \Code{func(d)}, where \Code{d} is calculated as the distance to \var{point}. The function must be applicable on arrays, so it should only include numerical operations and functions understood by the numpy module. By default, the distance \Code{d} is calculated in 3-D, but one can specify a limited set of axes to calculate a 2-D or 1-D distance.

This method is one of several mapping methods. See also \function{map()} and \function{map1()}.\\
Example: \Code{E.mapd(2,lambda d:sqrt(10**2-d**2),f.center(),[0,1])} maps E on a sphere with radius 10.
\end{methoddesc}

\begin{methoddesc}{replace}{i,j,other=None}
Replace the coordinates along the axes \var{i} by those along \var{j}.

\var{i} and \var{j} are lists of axis numbers.\\
\Code{replace ([0,1,2],[1,2,0])} will roll the axes by 1.\\
\Code{replace ([0,1],[1,0])} will swap axes 0 and 1.\\
An optionally third argument may specify another Formex to take the coordinates from. It should have the same dimensions.
\end{methoddesc}

\begin{methoddesc}{swapaxes}{i,j}
Swap coordinate axes \var{i} and \var{j}.
\end{methoddesc}

\begin{methoddesc}{circulize}{angle}
Transform a linear sector into a circular one.

A sector of the (0,1) plane with given angle, starting from the 0 axis,
is transformed as follows: points on the sector borders remain in
place. Points inside the sector are projected from the center on the
circle through the intersection points of the sector border axes and
the line through the point and perpendicular to the bisector of the
angle.
\end{methoddesc}

\begin{methoddesc}{circulize1}{}
Transforms the first octant of the 0-1 plane into 1/6 of a circle.

Points on the 0-axis keep their position. Lines parallel to the 1-axis
are transformed into circular arcs. The bisector of the first quadrant
is transformed in a straight line at an angle Pi/6.
This function is especially suited to create circular domains where
all bars have nearly same length. See the Diamatic example.
\end{methoddesc}

\begin{methoddesc}{shrink}{factor}
Shrinks each element with respect to its own center.

Each element is scaled with the given factor in a local coordinate
system with origin at the element center. The element center is the
mean of all its nodes.
The shrink operation is typically used (with a factor around 0.9) in
wireframe draw mode to show all elements disconnected. A factor above
1.0 will grow the elements.
\end{methoddesc}


\subsection{Topology changing transformations}

\begin{methoddesc}{replic}{n,step,dir=0}
Return a Formex with \var{n} replications in direction \var{dir} with \var{step}.

The original Formex is the first of the n replicas.
\end{methoddesc}

\begin{methoddesc}{replic2}{n1,n2,t1,t2,d1=0,d2=1,bias=0,taper=0}
Replicate in two directions.

\var{n1,n2} : number of replications with steps \var{t1,t2} in directions \var{d1,d2}.\\
\var{bias, taper} : extra step and extra number of generations in direction \var{d1} for each generation in direction \var{d2}.
\end{methoddesc}

\begin{methoddesc}{rosette}{n,angle,axis=2,point=[0.,0.,0.]}
Return a Formex with \var{n} rotational replications with angular step \var{angle} around an axis parallel with one of the coordinate axes going through the given point. \var{axis} is the number of the axis (0,1,2). \var{point} must be given as a list (or array) of three coordinates. The original Formex is the first of the n replicas.
\end{methoddesc}

\begin{methoddesc}{translatem}{*args}
Multiple subsequent translations in axis directions.

The argument \var{list} is a sequence of tuples \var{(axis, step)}. Thus \Code{translatem((0,x),(2,z),(1,y))} is equivalent to \Code{translate([x,y,z])}. This function is especially conveniant to translate in calculated directions.
\end{methoddesc}


\subsection{Write to file, read from file}


\begin{methoddesc}{write}{fil,sep=' '}
Write a Formex to file.

If \var{fil} is a string, a file with that name is opened. Else \var{fil} should
be an open file.
The Formex is then written to that file in a native format. It is advised, though not required, to use filenames ending in '.formex' for this purpose.
If \var{fil} is a string, the file is closed prior to returning.

If \var{sep} is specified, it will be used as a separator between subsequent coordinates. If an empty string is specified, the formex will be stored in a binary format. The default is to use an ASCII format with a single space as separator.
\end{methoddesc}


\begin{methoddesc}{read}{fil}
Read a Formex from a file in native format. 

This class method can be used to read back the data stored with the \Code{write(fil,sep)} method. \var{fil} is either a filename, or an open file.

There is no need to specify the separator that was used in the write operation: it will be detected from the file header.

Also, note the existence of a \var{readfile} function that can be used to read Formex data from a file that is not in native format. 

\classmethod
\end{methoddesc}


\subsection{Non-member functions}
The following functions operate on or return Formex objects, but are not part of the Formex class.\footnote{They might be implemented as class methods in future releases.}

\begin{funcdesc}{connect}{Flist,nodid=None,bias=None,loop=False}
Return a Formex which connects the formices in \var{Flist}.

\var{Flist} is a list of formices, \var{nodid} is an optional list of nod ids and \var{bias} is an optional list of element bias values. All lists should have the same length. The returned Formex has a plexitude equal to the number of formices in \var{Flist}. Each element of the Formex consist of a node from the corresponding element of each of the Formices in \var{Flist}. By default this will be the first node of that element, but a \var{nodid} list may be given to specify the node ids to be used for each of the formices. Finally, a list of bias values may be given to specify an offset in element number for the subsequent Formices.

If \var{loop} is False, the length of the Formex will be the minimum length of the formices in \var{Flist}, each minus its respective bias. By setting \var{loop} True however, each Formex will loop around when its end is encountered, and the length of the resulting Formex is the maximum length in \var{Flist}.
\end{funcdesc}


\begin{funcdesc}{interpolate}{F,G,div}
Create interpolations between two formices.

\var{F} and \var{G} are two Formices with the same shape.
\var{v} is a list of floating point values.
The result is the concatenation of the interpolations of \var{F} and \var{G} at all the values in \var{div}.

An interpolation of \var{F} and \var{G} at value \var{v} is a Formex \var{H} where each coordinate \var{Hijk} is obtained from  \Code{Hijk = Fijk + v * (Gijk-Fijk)}.
Thus, a Formex \Code{interpolate(F,G,[0.,0.5,1.0])} will contain all elements
of \var{F} and \var{G} and all elements with mean coordinates between those of \var{F} and \var{G}.

As a convenience, if an integer is specified for \var{div}, it is taken as a
number of division for the interval [0..1].
Thus, \Code{interpolate(F,G,n)} is equivalent with
\Code{interpolate(F,G,arange(0,n+1)/float(n))}
\end{funcdesc}


\begin{funcdesc}{divide}{F,div}
Divide a plex-2 Formex at the values in div.

Replaces each member of the Formex \var{F} by a sequence of members obtained
by dividing the Formex at the relative values specified in \var{div}. The values
should normally range from 0.0 to 1.0.
    
As a convenience, if an integer is specified for \var{div}, it is taken as a
number of divisions for the interval [0..1].

This function only works on plex-2 Formices (line segments).
\end{funcdesc}


\begin{funcdesc}{readfile}{file,sep=',',plexitude=1,dimension=3}
Read a Formex from file.

This convenience function uses the numpy fromfile function to read the coordinates of a Formex from file. 

\var{file} is either an open file object or a string with the name of the file to be read.
\var{sep} is the separator string between subsequent coordinates. There can be extra blanks around the separator, and the separator can be omitted at the end of line. If an empty string is specified, the file is read in binary mode.

The file is read as a single stream of coordinates; the arguments \var{plexitude} and \var{dimension} determine how these are structured into a Formex.
\var{plexitude} is the number of points that make up an element. The default is to return a plex-1 Formex (unconnected points).
\var{dimension} is the number of coordinates that make up a point (2 or 3). As always, the resulting Formex will be 3D.
The total number of coordinates on the file should be a multiple of \Code{plexitude * dimension}.
\end{funcdesc}


\begin{funcdesc}{bbox}{formexlist}
Computes the overall bounding box of a list of Formices.

The result has the same format as Formex class \Code{bbox()} method, but the resulting box encloses all the Formices in the list.
\end{funcdesc}



\subsection{Other functions}
The following functions are defined in the file formex.py, but do not depend on the Formex class. They are defined here because they are mainly supporting functions for the Formex class itself.


\begin{funcdesc}{sind}{arg}
    Return the sin of an angle in degrees.
\end{funcdesc}

\begin{funcdesc}{cosd}{arg}
    Return the cos of an angle in degrees.
\end{funcdesc}

\begin{funcdesc}{tand}{arg}
    Return the tan of an angle in degrees.
\end{funcdesc}

\begin{funcdesc}{length}{arg}
    Return the quadratic norm of a vector with all elements of arg.
\end{funcdesc}

\begin{funcdesc}{inside}{p,mi,ma}
    Return true if point p is inside bbox defined by points mi and ma
\end{funcdesc}

% \begin{funcdesc}{unique}{a}
%     Return the unique elements in an integer array.
% \end{funcdesc}

\begin{funcdesc}{isClose}{values,target,rtol=1.e-5,atol=1.e-8}
Return an array flagging the elements in values that are close to target.

\var{values} is a float array, \var{target} is a float value.
\var{values} and \var{target} should be broadcastable to the same shape.
    
The return value is a boolean array with shape of \var{values} flagging
the values that are close to target.
Two values \var{a} and \var{b}  are considered close if 
\Code{abs(a-b) < atol + rtol * abs(b)}
\end{funcdesc}

\begin{funcdesc}{vectorPairAreaNormals}{vec1,vec2}
Compute area of and normals on parallellograms formed by two vectors.

\var{vec1} and \var{vec2} are (n,3)-shaped arrays holding collections of vectors. 
The result is a tuple of two arrays:
\begin{itemize}
\item \var{area}, with shape (n): the area of the parallellogram formed by corresponding vectors of \var{vec1} and \var{vec2}.
\item \var{normal}, with shape (n,3): unit-length normal vectors to each pair of vectors. The positive direction of the normals is thus that a rotation of \var{vec1} to \var{vec2} corresponds to a positive rotation around the normal.
\end{itemize}
Both values are calculated from the cross prduct of \var{vec1} and \var{vec2}, which indeed results in \var{area} * \var{normal}.
\end{funcdesc}

\begin{funcdesc}{vectorPairArea}{vec1,vec2}
Compute the area of the parallellogram formed by two vectors.

This returns the first part of \Code{vectorPairAreaNormals(vec1,vec2)}.
\end{funcdesc}

\begin{funcdesc}{vectorPairNormals}{vec1,vec2,normalized=True}
Compute the normal vectors to pairs of two vectors.

With \Code{normalized=True}, this returns the second part of \Code{vectorPairAreaNormals(vec1,vec2)}.

With \Code{normalized=False}, returns unnormalized normal vectors. This does not use the \Code{vectorPairAreaNormals} function and is provided only to save computing time with very large arrays when normalization is not required. It is equivalent to \Code{cross(vec1,vec2)}.

\end{funcdesc}


\begin{funcdesc}{pattern}{s}
Return a line segment pattern created from a string.

This function creates a list of line segments where all nodes lie on the gridpoints of a regular grid with unit step.
The first point of the list is [0,0,0]. Each character from the given string is interpreted as a code specifying how to move to the next node.\\
Currently defined are the following codes:\\
0 = goto origin [0,0,0]\\
1..8 move in the x,y plane\\
9 remains at the same place\\
When looking at the plane with the x-axis to the right,\\
1 = East, 2 = North, 3 = West, 4 = South, 5 = NE, 6 = NW, 7 = SW, 8 = SE.\\
Adding 16 to the ordinal of the character causes an extra move of +1 in the z-direction. Adding 48 causes an extra move of -1. This means that 'ABCDEFGHI', resp. 'abcdefghi', correspond with '123456789' with an extra z +/-= 1. This gives the following schema:
\begin{verbatim}
                 z+=1             z unchanged            z -= 1
            
             F    B    E          6    2    5         f    b    e 
                  |                    |                   |     
                  |                    |                   |     
             C----I----A          3----9----1         c----i----a  
                  |                    |                   |     
                  |                    |                   |     
             G    D    H          7    4    8         g    d    h
\end{verbatim}             
The special character '\verb?\?' can be put before any character to make the move without making a connection. The effect of any other character is undefined. The resulting list is directly suited to initialize a Formex.
\end{funcdesc}


\begin{funcdesc}{translationVector}{dir,dist}
    Return a translation vector in direction dir over distance dist
\end{funcdesc}

\begin{funcdesc}{rotationMatrix}{angle,axis=None}
    Return a rotation matrix over angle, optionally around axis.

    The angle is specified in degrees.
    If axis==None (default), a 2x2 rotation matrix is returned.
    Else, axis should be one of [ 0,1,2] and specifies the rotation axis
    in a 3D world. A 3x3 rotation matrix is returned.
\end{funcdesc}
    

\begin{funcdesc}{rotationAboutMatrix}{angle,axis}
    Return a rotation matrix over angle around an axis thru the origin.

    The angle is specified in degrees.
    Axis is a list of three components specifying the axis.
    The result is a 3x3 rotation matrix in list format.
    Note that:
      rotationAboutMatrix(angle,[1,0,0]) == rotationMatrix(angle,0) 
      rotationAboutMatrix(angle,[0,1,0]) == rotationMatrix(angle,1) 
      rotationAboutMatrix(angle,[0,0,1]) == rotationMatrix(angle,2)
    but the latter functions are more efficient.
\end{funcdesc}
    


\begin{funcdesc}{equivalence}{x,nodesperbox=1,shift=0.5,rtol=1.e-5,atol=1.e-5}
    Finds (almost) identical nodes and returns a compressed list.

    The input x is an (nnod,3) array of nodal coordinates. This functions finds
    the nodes that are very close and replaces them with a single node.
    The return value is a tuple of two arrays: the remaining (nunique,3) nodal
    coordinates, and an integer (nnod) array holding an index in the unique
    coordinates array for each of the original nodes.

    The procedure works by first dividing the 3D space in a number of
    equally sized boxes, with a mean population of nodesperbox.
    The boxes are numbered in the 3 directions and a unique integer scalar
    is computed, that is then used to sort the nodes.
    Then only nodes inside the same box are compared on almost equal
    coordinates, using the numpy allclose() function. Two coordinates are
    considered close if they are within a relative tolerance rtol or absolute
    tolerance atol. See numpy for detail. The default atol is set larger than
    in numpy, because pyformex typically runs with single precision.
    Close nodes are replaced by a single one.

    Currently the procedure does not guarantee to find all close nodes:
    two close nodes might be in adjacent boxes. The performance hit for
    testing adjacent boxes is rather high, and the probability of separating
    two close nodes with the computed box limits is very small. Nevertheless
    we intend to access this problem by repeating the procedure with the
    boxes shifted in space.
\end{funcdesc}
    


\begin{funcdesc}{distanceFromPlane}{f,p,n}
    Return the distance of points f from the plane (p,n).

    f is an [...,3] array of coordinates.
    p is a point specified by 3 coordinates.
    n is the normal vector to a plane, specified by 3 components.

    The return value is a [...] shaped array with the distance of
    each point to the plane through p and having normal n.
    Distance values are positive if the point is on the side of the
    plane indicated by the positive normal.
\end{funcdesc}
    


\begin{funcdesc}{distanceFromLine}{f,p,q}
    Return the distance of points f from the line (p,q).

    f is an [...,3] array of coordinates.
    p and q are two points specified by 3 coordinates.

    The return value is a [...] shaped array with the distance of
    each point to the line through p and q.
    All distance values are positive or zero.
\end{funcdesc}
    


\begin{funcdesc}{distanceFromPoint}{f,p}
    Return the distance of points f from the point p.

    f is an [...,3] array of coordinates.
    p is a point specified by 3 coordinates.

    The return value is a [...] shaped array with the distance of
    each point to the line through p and q.
    All distance values are positive or zero.
\end{funcdesc}
    


\section{simple --- simple geometries}
{\label{sec:simple}
This module contains some functions, data and classes for generating Formices with simple geometries. 


\begin{funcdesc}{circle}{a1=1.,a2=2.,a3=360.}
Creates a Formex which is a polygonal approximation to a circle or arc.

All points generated by this function lie on a circle with unit radius at the origin in the x-y-plane.

\var{a1} (the dash angle) is the angle enclosed by the begin and end point of each line segment.\\
\var{a2} (the module angle) is the angle enclosed between the start points of two subsequent line segments.\\
\var{a3} (the arc angle) is the total angle enclosed between the first point of the first segment and the en point of the last segment.

All angles are given in degrees and are measured in the direction from x to y-axis. The first point of the first segment is always on the x-axis.

Remark that a1 == a2 produces a continues line, a1 < a2 gives a dashed line.
The default a3=360. produces a full circle; for a3 < 360, the result is an arc.
Large angle values result in polygones: circle(120,120) is an equilateral triangle and circle(60,60) is regular hexagone. The default values produces a dashed (near-)circle.

\end{funcdesc}

%%% Local Variables: 
%%% mode: latex
%%% TeX-master: "manual"
%%% End: 
